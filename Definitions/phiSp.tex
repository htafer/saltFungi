\documentclass[fontsize=10pt, paper=a4,fleqn, ]{wlscirep} 
\title{Genomic and transcriptomic analysis of the fungi
  Cladophialophora immunda exposed to toluene}
\author[1,*,+]{Barbara Blasi}
\author[1,+]{Hakim Tafer}
\author[1]{Ksenija Lopandic}
\author[1]{Katja Sterflinger-Gleixner}
\affil[1]{Affiliation, department, city, postcode, country}
\affil[2]{Affiliation, department, city, postcode, country}

\affil[*]{corresponding.author@email.example}

\affil[+]{these authors contributed equally to this work}

%%\usepackage[dvipsnames]{xcolor}  % Allows the definition of hex colors
%%\usepackage[bottom=10em]{geometry} % Reduces the whitespace at the bottom of the page so more text can fit
%%%\usepackage{fontspec}
%%\usepackage[english]{babel} % English language
%%\usepackage{lipsum} % Used for inserting dummy 'Lorem ipsum' text into the template
%%\usepackage[utf8]{inputenc} % Uses the utf8 input encoding
%%\usepackage[T1]{fontenc} % Use 8-bit encoding that has 256 glyphs
%%
%%\usepackage[osf]{mathpazo} % Palatino as the main font
%%\linespread{1.05}\selectfont % Palatino needs some extra spacing, here 5% extra
%%\usepackage[scaled=.88]{beramono} % Bera-Monospace
%%\usepackage[scaled=.86]{berasans} % Bera Sans-Serif
%%\usepackage{listings}
%%\usepackage{booktabs,array} % Packages for tables
%%
%%\usepackage{amsmath} % For typesetting math
\usepackage{graphicx} % Required for including images
\usepackage{url}
\usepackage{color}
\usepackage{tcolorbox}
\tcbuselibrary{minted,skins}
\usepackage{minted}
\usemintedstyle{native}

\definecolor{mintedbackground}{rgb}{0,0,0}
\newmintedfile[perlCode]{perl}{
  formatcom=\color{green},
  breaklines=true,
  bgcolor=mintedbackground,
  fontfamily=tt,
  linenos=true,
  numberblanklines=true,
  numbersep=12pt,
  numbersep=5pt,
  gobble=0,
  frame=leftline,
  framerule=0.4pt,
  framesep=2mm,
  funcnamehighlighting=true,
  tabsize=4,
  obeytabs=false,
  mathescape=false
  samepage=false, %with this setting you can force the list to appear on
  %the same page
  showspaces=false,
  showtabs =false,
  texcl=false,
  fontsize=\footnotesize
}

\newmintedfile[pythonCode]{python}{
  formatcom=\color{green},
  breaklines=true,
  bgcolor=mintedbackground,
  fontfamily=tt,
  linenos=true,
  numberblanklines=true,
  numbersep=12pt,
  numbersep=5pt,
  gobble=0,
  frame=leftline,
  framerule=0.4pt,
  framesep=2mm,
  funcnamehighlighting=true,
  tabsize=4,
  obeytabs=false,
  mathescape=false
  samepage=false, %with this setting you can force the list to appear on
  %the same page
  showspaces=false,
  showtabs =false,
  texcl=false,
  fontsize=\footnotesize
}

\newmintedfile[Rcode]{R}{
  formatcom=\color{green},
  breaklines=true,
  bgcolor=mintedbackground,
  fontfamily=tt,
  linenos=true,
  numberblanklines=true,
  numbersep=12pt,
  numbersep=5pt,
  gobble=0,
  frame=leftline,
  framerule=0.4pt,
  framesep=2mm,
  funcnamehighlighting=true,
  tabsize=4,
  obeytabs=false,
  mathescape=false
  samepage=false, %with this setting you can force the list to appear on
  %the same page
  showspaces=false,
  showtabs =false,
  texcl=false,
  fontsize=\footnotesize
}

\newmintedfile[bashCode]{bash}{
  formatcom=\color{green},
  breaklines=true,
  bgcolor=mintedbackground,
  fontfamily=tt,
  linenos=true,
  numberblanklines=true,
  numbersep=12pt,
  numbersep=5pt,
  gobble=0,
  frame=leftline,
  framerule=0.4pt,
  framesep=2mm,
  funcnamehighlighting=true,
  tabsize=4,
  obeytabs=false,
  mathescape=false
  samepage=false, %with this setting you can force the list to appear on
  %the same page
  showspaces=false,
  showtabs =false,
  texcl=true,
  fontsize=\footnotesize
}

\newmintedfile[xmlCode]{xml}{
  formatcom=\color{green},
  breaklines=true,
  bgcolor=mintedbackground,
  fontfamily=tt,
  linenos=true,
  numberblanklines=true,
  numbersep=12pt,
  numbersep=5pt,
  gobble=0,
  frame=leftline,
  framerule=0.4pt,
  framesep=2mm,
  funcnamehighlighting=true,
  tabsize=4,
  obeytabs=false,
  mathescape=false
  samepage=false, %with this setting you can force the list to appear on
  %the same page
  showspaces=false,
  showtabs =false,
  texcl=false,
  fontsize=\footnotesize
}



%%Conditional formatting in order to include / exclude code
%%show code
\newif\ifcode
\codefalse
%%\usepackage{etoolbox}
%%\usepackage[norule]{footmisc} % Removes the horizontal rule from footnotes
%%\usepackage{lastpage} % Counts the number of pages of the documen
%%\usepackage[T1]{fontenc}
%%\usepackage[]{minted}
%%\usepackage{tcolorbox}
%%\usepackage{lineno}
%%\usepackage{etoolbox}
%%\definecolor{titleblue}{rgb}{0.16,0.24,0.64} % Custom color for the title on the title page
%%\definecolor{linkcolor}{rgb}{0,0,0.42} % Custom color for links - dark blue at the moment
%%\addtokomafont{title}{\Huge\color{titleblue}} % Titles in custom blue color
%%\addtokomafont{chapter}{\color{OliveGreen}} % Lab dates in olive green
%%\addtokomafont{section}{\color{Sepia}} % Sections in sepia
%%\addtokomafont{pagehead}{\normalfont\sffamily\color{gray}} % Header text in gray and sans serif
%%\addtokomafont{caption}{\footnotesize\itshape} % Small italic font size for captions
%%\addtokomafont{captionlabel}{\upshape\bfseries} % Bold for caption labels
%%\addtokomafont{descriptionlabel}{\rmfamily}
%%\setcapwidth[r]{10cm} % Right align caption text
%%\setkomafont{footnote}{\sffamily} % Footnotes in sans serif
%%
%%\deffootnote[4cm]{4cm}{1em}{\textsuperscript{\thefootnotemark}} % Indent footnotes to line up with text
%%
%%\DeclareFixedFont{\textcap}{T1}{phv}{bx}{n}{1.5cm} % Font for main title: Helvetica 1.5 cm
%%\DeclareFixedFont{\textaut}{T1}{phv}{bx}{n}{0.8cm} % Font for author name: Helvetica 0.8 cm
%%
%%\usepackage[nouppercase,headsepline]{scrpage2} % Provides headers and footers configuration
%%\pagestyle{scrheadings} % Print the headers and footers on all pages
%%\clearscrheadfoot % Clean old definitions if they exist
%%
%%\automark[chapter]{chapter}
%%\ohead{\headmark} % Prints outer header
%%
%%\setlength{\headheight}{25pt} % Makes the header take up a bit of extra space for aesthetics
%%\setheadsepline{.4pt} % Creates a thin rule under the header
%%\addtokomafont{headsepline}{\color{lightgray}} % Colors the rule under the header light gray
%%
%%\ofoot[\normalfont\normalcolor{\thepage\ |\  \pageref{LastPage}}]{\normalfont\normalcolor{\thepage\ |\  \pageref{LastPage}}} % Creates an outer footer of: "current page | total pages"
%%
%%% These lines make it so each new lab day directly follows the previous one i.e. does not start on a new page - comment them out to separate lab days on new pages
%%\makeatletter
%%\patchcmd{\addchap}{\if@openright\cleardoublepage\else\clearpage\fi}{\par}{}{}
%%\makeatother
%%\renewcommand*{\chapterpagestyle}{scrheadings}
%%
%%% These lines make it so every figure and equation in the document is numbered consecutively rather than restarting at 1 for each lab day - comment them out to remove this behavior
%%\usepackage{chngcntr}
%%\counterwithout{figure}{labday}
%%\counterwithout{equation}{labday}
%%
%%% Hyperlink configuration
%%\usepackage[
%%    pdfauthor={}, % Your name for the author field in the PDF
%%    pdftitle={Laboratory Journal}, % PDF title
%%    pdfsubject={}, % PDF subject
%%    bookmarksopen=true,
%%    linktocpage=true,
%%    urlcolor=linkcolor, % Color of URLs
%%    citecolor=linkcolor, % Color of citations
%%    linkcolor=linkcolor, % Color of links to other pages/figures
%%    backref=page,
%%    pdfpagelabels=true,
%%    plainpages=false,
%%    colorlinks=true, % Turn off all coloring by changing this to false
%%    bookmarks=true,
%%    pdfview=FitB]{hyperref}
%%\usepackage[stretch=10]{microtype} % Slightly tweak font spacing for aesthetics
%%
%%%Set listings options
%%
%%\definecolor{mygreen}{rgb}{0,0.6,0}
%%\definecolor{mygray}{rgb}{0.5,0.5,0.5}
%%\definecolor{mymauve}{rgb}{0.58,0,0.82}
%%
%%\lstset{
%%  backgroundcolor=\color{white},   % choose the background color; you must add \usepackage{color} or \usepackage{xcolor}
%%  basicstyle=\footnotesize,        % the size of the fonts that are used for the code
%%  breakatwhitespace=false,         % sets if automatic breaks should only happen at whitespace
%%  breaklines=true,                 % sets automatic line breaking
%%  captionpos=b,                    % sets the caption-position to bottom
%%  commentstyle=\color{mygreen},    % comment style
%%  deletekeywords={...},            % if you want to delete keywords from the given language
%%  escapeinside={\%*}{*)},          % if you want to add LaTeX within your code
%%  extendedchars=true,              % lets you use non-ASCII characters; for 8-bits encodings only, does not work with UTF-8
%%  frame=single,                    % adds a frame around the code
%%  keepspaces=true,                 % keeps spaces in text, useful for keeping indentation of code (possibly needs columns=flexible)
%%  keywordstyle=\color{blue},       % keyword style
%%  language=Python,                 % the language of the code
%%  morekeywords={*,...},            % if you want to add more keywords to the set
%%  numbers=left,                    % where to put the line-numbers; possible values are (none, left, right)
%%  numbersep=5pt,                   % how far the line-numbers are from the code
%%  numberstyle=\tiny\color{mygray}, % the style that is used for the line-numbers
%%  rulecolor=\color{black},         % if not set, the frame-color may be changed on line-breaks within not-black text (e.g. comments (green here))
%%  showspaces=false,                % show spaces everywhere adding particular underscores; it overrides 'showstringspaces'
%%  showstringspaces=false,          % underline spaces within strings only
%%  showtabs=false,                  % show tabs within strings adding particular underscores
%%  stepnumber=2,                    % the step between two line-numbers. If it's 1, each line will be numbered
%%  stringstyle=\color{mymauve},     % string literal style
%%  tabsize=2,                       % sets default tabsize to 2 spaces
%%  title=\lstname                   % show the filename of files included with \lstinputlisting; also try caption instead of title
%%}
%%
%%
%%
%%
%%%\setlength\parindent{0pt} % Uncomment to remove all indentation from paragraphs
%%
%%%----------------------------------------------------------------------------------------
%%%	DEFINITION OF EXPERIMENTS
%%%----------------------------------------------------------------------------------------
%%
%%% Template: \newexperiment{<abbrev>}[<short form>]{<long form>}
%%% <abbrev> is the reference to use later in the .tex file in \experiment{}, the <short form> is only used in the table of contents and running title - it is optional, <long form> is what is printed in the lab book itself
%%
%%\newexperiment{example}[Example experiment]{This is an example experiment}
%%\newexperiment{example2}[Example experiment 2]{This is another example experiment}
%%\newexperiment{example3}[Example experiment 3]{This is yet another example experiment}
%%
%%\newsubexperiment{subexp_example}[Example sub-experiment]{This is an example sub-experiment}
%%\newsubexperiment{subexp_example2}[Example sub-experiment 2]{This is another example sub-experiment}
%%\newsubexperiment{subexp_example3}[Example sub-experiment 3]{This is yet another example sub-experiment}
%%
\newcommand{\E}[2]{${#1}\cdot10^{-{#2}}$}

%\newcommand{\textdegree}{$^o$}
%\newcommand{\textmu}{$\mu$}
%\newcommand{\textonehalf}{1/2}
\newcommand{\textlower}{$<$}
\newcommand{\TODO}[1]{\textbf{\color{red}{#1}}}
\newcommand{\exoDer}{\textit{Exophiala dermatitidis}}
\newcommand{\exoMes}{\textit{Exophiala mesophila}}
\newcommand{\exoSid}{\textit{Exophiala sideris}}
\newcommand{\exoSpi}{\textit{Exophiala spinifera}}
\newcommand{\claImm}{\textit{Cladophialophora immunda}}
\newcommand{\fonPed}{\textit{Fonsacaea pedrosi}}
\newcommand{\horWer}{\textit{Hortea werneckii}}
\newcommand{\canAlb}{\textit{Candida albicans}}
\newcommand{\aspNig}{\textit{Aspergillus niger}}
\newcommand{\aspNid}{\textit{Aspergillus nidulans}}
\newcommand{\aspFum}{\textit{Aspergillus fumigatus}}
\newcommand{\aspRub}{\textit{Aspergillus ruber}}
\newcommand{\penChr}{\textit{Penicillium chrysogenum}}
\newcommand{\debFab}{\textit{Debaryomyces fabryi}}
\newcommand{\debHan}{\textit{Debaryomyces hansenii}}
\newcommand{\walIch}{\textit{Walemia ichthyophaga}}
\newcommand{\walMel}{\textit{Walemia mellicola}}
\newcommand{\canPar}{\textit{Candida parapsilosis}}
\newcommand{\cocImm}{\textit{Coccidioides immitis}}
\newcommand{\triRub}{\textit{Trichophyton rubrum}}
\newcommand{\thiTer}{\textit{Thielavia terrestris}}
\newcommand{\mycThe}{\textit{Myceliophthora thermophila}}
\newcommand{\neuCra}{\textit{Neurospora crassa}}
\newcommand{\sacCer}{\textit{Saccharomyces cerevisiae}}
\newcommand{\schPom}{\textit{Schizosaccharomyces pombe}}
\newcommand{\cryNeo}{\textit{Cryptococcus neoformans}}
\newcommand{\micCan}{\textit{Microsporum canis}}
\newcommand{\onyCor}{\textit{Onygena corvina}}
\newcommand{\blaDer}{\textit{Blastomyces dermatitidis}}
\newcommand{\parBra}{\textit{Paracoccidioides brasiliensis}}
\newcommand{\cocPos}{\textit{Coccidioides posadasii}}
\newcommand{\toxGon}{\textit{Toxoplasma gondii}}
\newcommand{\fusOxy}{\textit{Fusarium oxysporum}}
\newcommand{\fusSol}{\textit{Fusarium solani}}
\newcommand{\tryRub}{\textit{Fusarium solani}}
\newcommand{\clado}{\textit{Cladophialophora}}
\newcommand{\attspe}{\textit{Attalea speciosa}}
\newcommand{\claban}{\textit{Cladophialophora bantiana}}
\newcommand{\clasat}{\textit{Cladophialophora saturnica}}
\newcommand{\clapsa}{\textit{Cladophialophora psammophila}}
\newcommand{\clasph}{\textit{Cladosporium sphaerospermum}}
\newcommand{\claBan}{\textit{Cladophialophora bantiana}}
\newcommand{\fusGra}{\textit{Fusarium graminearum}}
\newcommand{\fusSpo}{\textit{Fusarium sporotrichoides}}
\newcommand{\triAru}{\textit{Trichoderma arundinaceum}}
\newcommand{\triBre}{\textit{Trichoderma brevicompactum}}
\newcommand{\phiSp}{\textit{Phialosimplex sp. HF37}}
\newcommand{\phiScl}{\textit{Phialosimplex sclerolatus}}
\newcommand{\staCha}{\textit{Stachybotrys chartarum}}
\newcommand{\staChl}{\textit{Stachybotrys chloronata}}
\newcommand{\exo}{\textit{Exophiala}}
\newcommand{\herpo}{\textit{Herpotrichiellaceae}}
\newcommand{\skin}{skin}
\newcommand{\tol}{\textit{toluene}}
\newcommand{\glu}{\textit{glucose}}
%\newcommand{\textlower}{$<$}




%----------------------------------------------------------------------------------------

\begin{abstract}
\TODO{200 Words}
\end{abstract}

\begin{document}
\flushbottom
\maketitle
\thispagestyle{empty}
\noindent Please note: Abbreviations should be introduced at the first mention in the main text - no abbreviations lists. Suggested structure of main text (not enforced) is provided below.
\section{Introduction*}


%%Compared to bacteria tolerance to dry conditions 
%%Nevertheless it is an opportunistic human pathogen able to cause and its direct application is at the moment impossible. 
%%At the molecular level, little is known about this fungus when occupying its ecological niches and further investigations are required
%%Il tutto conferisce l abilita di colonizzare vertebrates
%%From the paper Indoor wet cells harbour melanized agents of cutaneous
%%Infection li an and de hoog 
%%Exophiala lecanii-corni has been used for biofiltration
%%and is able to remove toluene from air streams with a very
%%high elimination capacity [34]. Gunsch et al. [35] proved
%%that the gene ElHDO (homogentisate-1,2-dioxygenase) was
%%involved in the fungus’ ability to degradate VOCs. The
%%species has also been reported from systematic human infections,
%%e.g., respiratory tract, stomach and intestines, and was
%%also involved in cutaneous and subcutaneous cases [5].

\section{Method*}
\subsection{Fungal growth conditions}

\subsection{DNA isolation and Sequencing}

\subsection{Bioinformatics}
\subsubsection{Genome Annotation}
The deNovo ncRNAs annotation was done with
Infernal~\cite{Nawrocki2013} and the Rfam
database~\cite{Nawrocki2014}, tRNAscan~\cite{Schattner:05} and
SNOSTRIP~\cite{Bartschat2014}. Overlapping ncRNAs annotations were
merged manually. 

\TODO{Protein Coding}
%%Fusion transcripts were detected with the STAR-Fusion pipeline 
%%~\url{https://github.com/STAR-Fusion/STAR-Fusion} while circular RNAs
%%were detected by looking directly at the read mapping patterns. In
%%order to be reported, circular RNAs had to be supported by at least one
%%split read in two replicates of the same experimental
%%condition. Fusion transcripts were reported if they appeared in at
%%least three replicates. 


\subsubsection{Comparative Genomics}
Gene sequences and annotations for \aspNig, \aspNid, \aspFum, \claBan,
\exoDer, \fusOxy, \fusSol, \neuCra, \sacCer, {\schPom} and {\triRub} were
fetched from ensembl and ncbi. For {\exoMes}, our own genome assembly
and annotation were used~\cite{Tafer2015a}.
\ifcode
\begin{listing}[h]
  \perlCode{/media/work/genomes/claImm/comparativeGenomicsAnalysis/biomartWebService.pl}
  \label{code:fAperl}
  \caption{Perl used to fetch the protein functional annotation of 6/10 genomes.}
\end{listing}
\begin{listing}[h]
  \xmlCode{/media/work/genomes/claImm/comparativeGenomicsAnalysis/fetch.xml}
  \label{code:fAxml}
  \caption{xml used to fetch the protein functional annotation of 6/10 genomes.}
\end{listing}
\begin{listing}[h]
  \mint[breaklines,bgcolor=black,formatcom=\color{white},fontsize=\footnotesize]{shell}{parallel -j 8 "perl ./biomartWebService.pl -s {} -x fetch.xml > {}.csv" ::: anidulans aniger afumigatus ncrassa scerevisiae spombe}
  \label{code:fAbash}
  \caption{shell command used to fetch the protein functional annotation of 6/9 genomes.}
\end{listing}
\fi

The protein-coding genes of all studied species were functionally
annotated with a local install of
interproscan-5.19-58.0~\cite{Jones2014}. Homologies with   the
transporter classification database (TCDB)~\cite{Saier2016}, the
peptidase datavbase (MEROPS)~\cite{Rawlings2014} and the
carbohydrate-active enzymes database (CAZY)~\cite{Cantarel2009} were
assessed with blastp~\cite{Altschul:97b}(E-value $<$ 1e-3). Genes
homology was assessed with
Proteinortho-5.13~\cite{Lechner2011}. Pairwise generation of the
chains of synthenic genes for each genomes pair was done with
DAGchainer~\cite{Haas2004}. Horizontally transfered genes were
detected with the latest version of HGTFinder
(2016)~\cite{Nguyen2015}.


\ifcode
\begin{listing}[h]
    \mint[breaklines,bgcolor=black,formatcom=\color{white},fontsize=\footnotesize,formatcom=\color{green}]{shell}
         {
    snakemake -d `pwd` -s `pwd`/functionalAnnotation.Snakemake --stats
    snakemake.stats -j 100 --cluster 'qsub {params.cluster}'
    afumigatus.merops anidulans.merops aniger.merops foxysporum.merops
    fsolani.merops ncrassa.merops scerevisiae.merops spombe.merops
    trubrum.merops --printshellcmds}
  
  \mint[breaklines,bgcolor=black,formatcom=\color{white},fontsize=\footnotesize,formatcom=\color{green}]{shell}{
    snakemake -d `pwd` -s `pwd`/functionalAnnotation.Snakemake --stats
    snakemake.stats -j 100 --cluster 'qsub {params.cluster}'
    afumigatus.tsv anidulans.tsv aniger.tsv foxysporum.tsv
    fsolani.tsv ncrassa.tsv scerevisiae.tsv spombe.tsv
    trubrum.tsv --printshellcmds}
  
  \mint[breaklines,bgcolor=black,formatcom=\color{white},fontsize=\footnotesize,formatcom=\color{green}]{shell}{
    snakemake -d `pwd` -s `pwd`/functionalAnnotation.Snakemake --stats
    snakemake.stats -j 100 --cluster 'qsub {params.cluster}'
    afumigatus.cazy anidulans.cazy aniger.cazy foxysporum.cazy
    fsolani.cazy ncrassa.cazy scerevisiae.cazy spombe.cazy
    trubrum.cazy --printshellcmds}

  \label{code:functional annotation}
  \caption{command to functionally annotate proteins.}
\end{listing}
\fi




%%\ifcode
%%\begin{listing}[ht]
%%  \perlCode{./comparativeGenomicsAnalysis/perlRestClient.pl}
%%  \label{code:restPerl}
%%  \caption{Perl Rest client used to fetch the protein functional annotation of 3/9 genomes.}
%%\end{listing}
%%\begin{listing}[ht]
%%  \mint[breaklines,bgcolor=black,formatcom=\color{white},fontsize=\footnotesize]{bash}{
%%    perl ./perlRestClient.pl -u 'http://rest.ensemblgenomes.org/lookup/genome/Trichophyton_rubrum_cbs_118892?level=protein_feature' > tRubrum.csv
%%
%%    perl ./perlRestClient.pl -u 'http://rest.ensemblgenomes.org/lookup/genome/Fusarium_solani?level=protein_feature' > fSolati.csv
%%
%%    perl ./perlRestClient.pl -u 'http://rest.ensemblgenomes.org/lookup/genome/Fusarium_oxysporum?level=protein_feature' > fOxysporum.csv
%%
%%}
%%  \label{code:restBash}
%%  \caption{shell command used to fetch the protein functional
%%    annotation of 3/9 genomes with the REST client.}
%%\end{listing}
%%\fi


%%#!We need to do the following in order to set transcript and gene name to the same.
%%#!After that We will switch to transcript based differential expression
%%cat claImm.allProteinNewProteinLocus.gtf | perl -lane 'printf join("\t",@F[0..8]); print " ",$F[9]," ",$F[10]," ",$F[9]'


%%\subsubsection{Data upload into the genome browser}
%%A local install of the  genome browser in a box was populated with the
%%genome sequence, the protein annotation, the lncRNA annotation, the
%%transcript of unknown function, Rfam, RNAz, RNAcode, tRNA, circular RNAs and the
%%PHAST score of the conserved regions.
%%
%%\ifcode
%%\begin{listing}[ht]
%%  \bashCode{/media/work/genomes/claImm/annotation/convertToBB.sh}
%%  \label{code:uploadAnnotation}
%%  \caption{command to upload annotation to convert gtf to bb}
%%\end{listing}
%%\fi

\section{Results*}
\subsection{DNA Sequencing of the two \textit{Phialophora} species}

\subsection{Highly Complete Genome and Gene Set}
The genome assemblies of {\phiSp} and {\phiScl} used for this study
were seqeunced on the Ion PGM (See Methods). The {\phiSp} genome has a
size of 21.9Mb and contains a total of 8895 protein-coding
loci. {\phiScl} has a 27$\%$ larger genome (27.9Mb) with 27$\%$ more
coding-genes (11307) compared to {\phiSp}.
\subsection{Genes conservation and gene family evolution}
The gene conservation pattern between {\phiSp}, {\phiScl} and a set of
halophilic and halotolerant fungi listed in Table~\ref{tab:species}
was studied with Proteinortho~\cite{Lechner2011} (See Methods). 
\begin{table}
  \begin{center}
    \small
    \begin{tabular}{|l|l|p{2cm}|l|p{5cm}|}
      \hline
      Species   & Salt concentration & Type & Reference & Osmoadaptation \\
      \hline
      {\phiSp}  & 10-30$\%$,\TODO{optimal at 15$\%$} & Halophilic &
      This publication &\\ 
      {\phiScl} & 0-10$\%$, optimal at 10$\%$ & Halotolerant & This publication &\\
      {\aspRub} & $>$10$\%$,  optimal at 18$\%$&  Halophilic
      &\cite{Kis2014}&       expression of ion transporter, higher
      proportion of acidic amino acids gene duplication cell wall restructuring
      Glycerol\\
      {\penChr} & 0-18$\%$, optimal at 10$\%$ & Halotolerant
      &\cite{Attaby2001}& Glycerol \\
      {\horWer} & 0-32$\%$, optimal 3-9$\%$ & Halophile   &
      \cite{Plemenitas2015}& Polyols accumulation, alkali metal
      transport , K$^+$/Na$^+$, cell wall restructuring\\
      {\canPar} & 0-12$\%$, optimal 0 $\%$ & Halotolerant, xerotolerant & \cite{Krauke2010}&\\
      {\debFab} & 0-16$\%$, optimal 0 $\%$ & Halotolerant &
      \cite{Michan2013}& Polyols accumulation\\
      {\debHan} & 0-24$\%$, optimal 0 $\%$ & Halotolerant & \cite{Michan2013}& Polyols accumulation, sodium resistance\\
      {\sacCer} & 0-8$\%$, optimal 0 $\%$ & Control & \cite{Lages1999}&\\
      {\schPom} & 0-5$\%$, optimal 0 $\%$ & Control & \cite{Lages1999}&\\
      {\walIch} & >8 $\%$, optimal 18 $\%$ & Halophile &
      ~\cite{Zajc2013} & polyols accumulation, transport of alkali metal
      ions, hydrophobins\\
      {\walMel} & 0-27$\%$, optimal 0 $\%$ & Halotolerant,
      xerotolerant & ~\cite{Kuncic2010} & Cell wall thickening\\
      \hline
    \end{tabular}
    \caption{\label{tab:species} List of genomes compared in this
      study. For each species, the name, salt tolerance, type of
      tolerance, osmoadaptation and reference are listed}
    
  \end{center}
\end{table}

\ifcode
#check comparativeGenomics.snakemake
\end{listing}
\fi

%%%%%%%%%%%%%%%%%%%%%%%%%%%%%%%%%%%%%%%%%%%%%%%%%%%%%%%%%%%%%%%%%%%%%%%%%%%%%%%%
%
% paralogs comparison
%
%%%%%%%%%%%%%%%%%%%%%%%%%%%%%%%%%%%%%%%%%%%%%%%%%%%%%%%%%%%%%%%%%%%%%%%%%%%%%%%%


Proteinortho identified 1292 groups of ortholog genes present in all
12 species, among which 202 represented one-to-one relationships. In
{\phiSp} this corresponds to 1364 genes. 5819 genes, belonging to 5313
ortholog groups, were found in {\phiSp} and at least one other specie,
while 1712 genes were specific to this fungus. Among thos 1712 genes,
45 had at least 2 paralogues. For {\phiScl} 6929
genes belonging to 6029 ortholog groups are found in at least one
other specie, while 3003 genes where species-specific (See
Figure~\ref{fig:treeEnrichment}). The protein sequences of the 202
one-to-one ortholog groups  were used to construct the phylogenetic
tree shown in Figure~\ref{fig:treeEnrichment}. The tree corresponds to
the expected taxonomy, with {\phiSp} and {\phiScl} clustering and the two
wallemia species being on another branch than the other 10
ascomycetes. The functional enrichment of the 22 paralog groups
specific to {\phiScl}, corresponding to 45 genes, indicates that among
other integrase (IPR001584, fdr=\E{2.44}{7}), DUF4238 (IPR025332,
fdr=\E{1.4}{4}), fungal transcription factor (IPR021858,
fdr=\E{1.75}{4}), DUF2382 (IPR019060, \E{1.79}{4}), DUF2235
(IPR018712, \E{2.5}{3}) are overreprsented. Table~\ref{tab:over}
contains a list of overrepresented interproscan (IPR) and gene
ontology (GO) terms with an fdr$<$0.001.

\begin{table}
  \begin{center}
    \begin{tabular}{|l|l|l|}
    \hline
    Term & FDR & Description\\
    \hline
    "IPR001584"&Integrase catalytic core				        &\E{2.44}{7}\\
    "IPR012337"&Ribonuclease H-like domain				       &\E{1.31}{4}\\
    "IPR025332"&Protein of unknown function DUF4238			       &\E{1.40}{4}\\
    "IPR021858"&Protein of unknown function DUF3468			       &\E{1.75}{4}\\
    "IPR010921"&Trp repressor/replication initiator			       &\E{1.79}{4}\\
    "IPR019060"&Domain of unknown function DUF2382			       &\E{1.79}{4}\\
    "IPR025161"&Putative transposase IS4/IS5 family			       &\E{1.79}{4}\\
    "IPR018712"&Domain of unknown function DUF2235			       &\E{2.51}{4}\\
    "IPR000525"&Initiator Rep protein					       &\E{2.51}{4}\\
    "IPR002514"&Transposase IS3/IS911family				       &\E{2.51}{4}\\
    "IPR004276"&Glycosyltransferase family 28 N-terminal domain		       &\E{3.79}{4}\\
    "IPR010656"&TRAP C4-dicarboxylate transport system permease DctM subunit    &\E{8.91}{4}\\
    "IPR002213"&UDP-glucuronosyl/UDP-glucosyltransferase                        &\E{8.91}{4}\\ 
    \hline
    GO:0015074 & DNA integration    &\E{4.15}{8}\\
    GO:0070085 & lipid glycosylation&\E{4.41}{4}\\
    GP:0032196 & transposition      &\E{4.41}{4}\\
    \hline
  \end{tabular}
    
  \caption{\label{tab:over} Overrepresented IPR and GO terms in the set
    of {\phiSp}  specific paralogs with p-value $<$ 0.001. Only the most
    significant GO terms, in terms of fdr, for each Revigo categories
   are shown.}
  \end{center}
\end{table}


\begin{figure}[!h]
  \centering
  \includegraphics[width=\linewidth]{./Figure1.pdf}
  \caption{\label{fig:gainLossTree} Ultrametric tree with estimated
    divergence time derived from the one-to-one ortholog multiple
    sequences alignment and the divergence time between {\sacCer} and
    {\walIch}. Gain and loss of families are represented by barplots
    in the internal nodes. The \textbf{middle} represent the gain and
    losses in the leaf nodes. The color correspond to the family
    category while the shade is proportional to the magnitude in the
    number of changes. \textbf{r.h.s} The ortholog conservation for
    all 12  species studied is shown. Core genes, i.e. genes found in
    all genomes, are shown in dark blue. The number of conserved
    genes, i.e. genes found in at least two species, is shown in blue,
    while species specific genes are shown in light-blue.
  } 
\end{figure}




\ifcode
\begin{listing}
  \mint[breaklines,bgcolor=black,formatcom=\color{white},fontsize=\footnotesize]{shell}{
    cat myproject.proteinortho | perl -lane 'if($F[0]==12 &&
    ($F[1]==13 || $F[1]==12) && $F[2]==1){print;}'  | grep ,HOR  | sed -r 's/,HOR[^\t]+//g' > oneToOne.csv
    cat oneToOne.csv | perl ./generateAlignmentInput.pl -d . > oneToOneHomolog.phy 
    /home/htafer/bin/iqtree-omp-1.4.4-Linux/bin/iqtree-omp -s oneToOneHomolog.phy -m TEST -alrt 1000 -bb 1000 -nt 3
    #Tree was then created with evolgenius
  }
  \label{code:generatePhyloTree}
  \caption{shell command used to compute phylogenetic tree}
\end{listing}
\fi



\ifcode
\begin{listing}
  \mint[breaklines,bgcolor=black,formatcom=\color{white},fontsize=\footnotesize]{shell}
       {
         for i in `cat phiSp.merged.tsv | cut -f 4 | sort -u`; do printf "ANN\tIDs\n" > phiSp.ann.$i; grep $i phiSp.merged.tsv | perl -lane 'print $F[4]," ",$F[0]' | sort -k 1,1 -k 2,2 |  sort -u >> phiSp.ann.$i; done

    
    printf "IDs IEA Gene\n" > phiSp.ann.GO; 
    grep -P  "((GO:\d+)\|?)+" phiSp.merged.tsv | sed -r "s/^([^\t]+)\t+.+\t((GO:.+\|?)+).+/\1\t\2/" | cut -f 1,2 | sort -u | perl -lane 'foreach $GO (split(/\|/,$F[1])) {print $GO," IEA ",$F[0]};' | sort -k 1,1 -k 3,3 | uniq >> phiSp.ann.GO 

    cat phiSp.merged.tsv | grep ": " | grep -v  GO | sed -r 's/:\s+/:/g' | cut -f 1,15 | perl -lane '@array = split(/\|/,$F[1]); foreach my $GO (@array){print $GO," ",$F[0];}' | sed -r 's/\+[^ ]+/ /' | sort -k 1,1 -k 3,3  | uniq | sed -r 's/:/ /' | sed -r 's/KEGG /KEGG ko/' | sed -r s'/\s+/ /g' > pathways 
    
    for i in `cut -f 1 -d  ' ' pathways | sort -u`; do printf "ANN\tIDs\n" > phiSp.ann.${i}; grep ${i} pathways | cut -f 2,3 -d ' '| sort -k 1,1 -k 2,2 | uniq >> phiSp.ann.${i}; done 

    printf "ANN IDs\n" > phiSp.ann.IPR; grep IPR phiSp.merged.tsv | cut -f 1,12 | sort -k 1,1 -k 2,2 | uniq | perl -lane 'printf "$F[1] $F[0]\n"'>> phiSp.ann.IPR
    
    }
  \label{code:productionFileEnrichment}
  \caption{shell command used to generate the functional enrichment
    annotation files}
\end{listing}
\fi
\ifcode
\begin{listing}
  \mint[breaklines,bgcolor=black,formatcom=\color{white},fontsize=\footnotesize]{shell}{
    parallel -j 8 'j=`echo {1} | sed -r "s/phiSp.ann.//"`; enrichmentStat.R -b {1} -d ./paralogUniqToPhiSp.list -t $j -D /media/work/share/database/testDictionary.dat ;' ::: phiSp.ann.*
    
    cut -f 2,9 -d "#" paralogUniqToPhiSp.list.GO.csv | sed -r 's/\"//g' | sed -r 's/#/ /g' | sed -r 's/,/ /g' > paralogUniqToPhiSp.list.GO.csv.pV
    
    for i in *.pV; do 
     	 j=`echo $i`;
     	 echo $j;
     	 revigo.pl -f $i -t BP | sed -r "s/REVIGO Gene Ontology treemap/$j/" | sed -r "s/revigo_treemap.pdf/$j.tableBP.pdf/" > $i.tableBP
         csplit $i.tableBP /names\(stuff\)/+1  ; cat xx00 > $i.pieBP;
         cat ~/share/database/pie.chart.R | sed -r "s/FILENAME/\"$i.pieBP.pdf\"/" >> $i.pieBP;
         revigo.pl -f $i -t CC | sed -r "s/REVIGO Gene Ontology treemap/$j/" | sed -r "s/revigo_treemap.pdf/$j.tableCC.pdf/" > $i.tableCC;
     	 csplit $i.tableCC /names\(stuff\)/+1  ; cat xx00 > $i.pieCC;
         cat ~/share/database/pie.chart.R | sed -r "s/FILENAME/\"$i.pieCC.pdf\"/" >> $i.pieCC;    
     	 revigo.pl -f $i -t MF | sed -r "s/REVIGO Gene Ontology treemap/$j/" | sed -r "s/revigo_treemap.pdf/$j.tableMF.pdf/" > $i.tableMF;
     	 csplit $i.tableMF /names\(stuff\)/+1  ; cat xx00 > $i.pieMF;
     	 cat ~/share/database/pie.chart.R | sed -r "s/FILENAME/\"$i.pieMF.pdf\"/" >> $i.pieMF;
     done;
     
          for i in paralogUniqToPhiSp*.table??; do cat -A $i | grep -Po "(\"(G|t)[^^]+)" | sed -r 's/(\(|\)|\"|,$)//g' | sed -r 's/, / /g' | sed -r 's/,/#/g' > $i.summary.csv; done


    summarize.pl -s 'paralogUniqToPhiSp.list.(Pfam|Merops|Cazy|GO)' -d . -q /media/work/share/database/materialPalette.csv -P 0.001 -N 2 -r 0 -D '#' > stuff
    
    fingerPrint.R -f ./stuff -o test.pdf -c 2 
    
  }
  \label{code:enrichedTerm}
  \caption{Command used to generate the enriched terms for each
    annotation catergory and barplot for some select categories}
\end{listing}
\fi
\ifcode
\begin{listing}
  \mint[breaklines,bgcolor=black,formatcom=\color{white},fontsize=\footnotesize]{shell}{
    for i in *.pV; do
    j=`echo $i`;
    echo $j
    revigo.pl -f $i -t BP | sed -r "s/REVIGO Gene Ontology treemap/${j}/" | sed -r "s/revigo_treemap.pdf/${j}.tableBP.pdf/" > $i.tableBP
    csplit ${i}.tableBP /names\(stuff\)/+1  ; cat xx00 > ${i}.pieBP;
    cat ~/share/database/pie.chart.R | sed -r "s/FILENAME/\"${i}.pieBP.pdf\"/" >> ${i}.pieBP; 
    
    revigo.pl -f $i -t MF | sed -r "s/REVIGO Gene Ontology treemap/${j}/" | sed -r "s/revigo_treemap.pdf/${j}.tableCC.pdf/" > $i.tableMF
    csplit ${i}.tableMF /names\(stuff\)/+1  ; cat xx00 > ${i}.pieMF;
    cat ~/share/database/pie.chart.R | sed -r "s/FILENAME/\"${i}.pieMF.pdf\"/" >> ${i}.pieMF; 

    revigo.pl -f $i -t CC | sed -r "s/REVIGO Gene Ontology treemap/${j}/" | sed -r "s/revigo_treemap.pdf/${j}.tableMF.pdf/" > $i.tableCC
    csplit ${i}.tableCC /names\(stuff\)/+1  ; cat xx00 > ${i}.pieCC;
    cat ~/share/database/pie.chart.R | sed -r "s/FILENAME/\"${i}.pieCC.pdf\"/" >> ${i}.pieCC; 
    done

    for i in *.table??; do cat -A $i | grep -Po "\"(G|term)[^^]+" | sed -r 's/(\(|\)|\"|,$)//g' | sed -r 's/,/#/g' > $i.summary.csv; done
  }
  \label{code:enrichedTerm}
  \caption{Command used to generate the enriched terms for each
    annotation catergory and barplot for some select categories}
\end{listing}
\fi

%%%%%%%%%%%%%%%%%%%%%%%%%%%%%%%%%%%%%%%%%%%%%%%%%%%%%%%%%%%%%%%%%%%%%%%%%%%%%%%%
% 
% Full genome comparison    
%
%%%%%%%%%%%%%%%%%%%%%%%%%%%%%%%%%%%%%%%%%%%%%%%%%%%%%%%%%%%%%%%%%%%%%%%%%%%%%%%%

The enrichment or depletion of functional annotations and 
in {\phiSp} against the other 12 genomes was studied. Compared to
{\phiScl}, no gene family is significantly enriched in {\phiSp}. Inversely
{\phiSp} is significantly depleted in transporters belonging to the
facilitator superfamily transporters~\TODO{Was seen in bacteria check reference} (IPR020846~\E{1.47}{4}, TCDB
2.A.1~\E{8.44}{6}),  aminoglycoside phosphotransferase
(PF01636~\E{1.25}{5},IPR002575~\E{1.47}{4}) and sodium-dependent phosphate
transporter (PTHR11662~\E{5.30}{3}). 

Compared to {\sacCer} and {\schPom}, {\phiSp} is enriched in three
families of transporters: MFS (TCDB 2.A.1 \E{7.69}{9}), Fatty acid
group translocation (TCDB 4.C.1 \E{5.18}{4}) and type IV secretory
pathways (TCDB 3.A.7 \E{9.33}{3}). Protein domains were enriched among
others in Cytochrome P450 (IPR001128\E{1.21}{14}), MFS transporters
(PS50850\E{4.32}{8}), fungal specific transcription factor domain
(PF11951\E{4.86}{8}), Glycoside hydrolase (IPR017853, \E{6.94}{6}),
aromatic ring hydroxylase (PR00420,\E{9.66}{6}), short chain
dehydrogenase (IPR020904\E{9.89}{4}) and glucose-methanol-choline
oxidoreductase (PF05199 \E{1.87}{3}) \TODO{MICHT BE INVOLVED IN
    OSMOTOLERANCE}. 

A full list of enriched terms can be found in Supplementary Table S1.


The enrichment of interproscan, gene ontology and transporter terms in
{\phiSp} compared to the each species as well as against the
halophile, halotolerant and control groups was computed and is
graphically represented in figure~\ref{fig:speciesEnrichment} and
Supplementary Table S1. Compared to {\phiScl} and {\penChr}, the
halophile {\phiSp} is depleted in genes belonging to the major
facilitator superfamily (MFS), which facilitate the transport of small
solutes, (TCDB:2A1, IPR011701, IPR020846), but is enriched compared to {\sacCer}
and {\schPom}. A similar enrichment/depletion pattern is seen for
terms related to the regulation of transcription (GO:0006355, GO:0000981,
IPR001138) where {\phiSp} is depleted compared to {\aspRub} and
{\penChr} but enriched compared to {\horWer}. {\phiScl} is enriched in
terms related to oxidation-reduction and hydrolase compared to the studied
saccharomycetales, {\schPom} and the two \textit{Wallemia} species
(GO:055114, IPR001128, IPR002401, GO:0004353, IPR029058).
{\phiSp} is also depleted in protein kinase (IPR011009) and
transcription factors (IPR001138,fdr=\E{}{}) when compared to
{\phiScl} and {\penChr}, something that is also seen when
comparing halophile and halotolerant species(\E{3.13}{5}).

Halophile and Halotolerant fungi are enriched in genes related to
transmembrane transport (IPR020846), oxidoreductase (IPR001128),
NAD(P)-binding domain (IPR016040), fungal transcription factor
(IPR007219), dehydrogenase (IPR002347) ad glucose-methanol-choline
oxidoreductes (IPR000172, IPR007867). They are depleted in genes
related to integrase (IPR001584), reverse transcriptase (IPR013103)
and aspartic peptidase (IPR001969).






\begin{figure}[!h]
  \centering
  \includegraphics[width=0.8\linewidth,angle=90]{./dataMatrixGOIPTC.pdf}
  \caption{\label{fig:speciesEnrichment} 
    Functional enrichment of {\phiSp} vs. all other genomes, the
    control, halophile and halotolerant groups of
    species. Halotolerant vs Halophile, Halophile vs Control and
    Halophile vs Halotolerant are also represented. The blue squares
    represent underrepresented categories in the reference genome,
    while the green squares represent overrepresented categories. The
    shades are proportional to $-log_{10}(fdr)$ for the overrepresented
    and $log_{10}(fdr)$ for the underrepresented categories.
  } 
\end{figure}






\ifcode
\begin{listing}
  \mint[breaklines,bgcolor=black,formatcom=\color{white},fontsize=\footnotesize]{shell}{
    #in /media/work/genomes/phiSp/comparativeGenomics/proteinOrtho
    #Generate over under representation data for phiSp vs halophile and halotolerant
    parallel -j 8 "perl ./getOverUnderFuncAnno2.pl -d . -s {1} -b {2}"
    ::: phiSp ::: phiScl aspRub penChr horWer canPar debFab debHan
    sacCer schPom walIch walMel aspRub,horWer,walIch
    phiScl,penChr,canPar,debFab,debHan,walMel sacCer,schPom
    
    #Do the same for halotolerant, halophile 
    perl ./getOverUnderFuncAnno2.pl -d . -s phiSp,aspRub,horWer,walIch
    -b phiScl,penChr,canPar,debFab,debHan,walMel 
    perl ./getOverUnderFuncAnno2.pl -d . -s phiSp,aspRub,horWer,walIch -b  sacCer,schPom
    perl ./getOverUnderFuncAnno2.pl -d . -s phiScl,penChr,canPar,debFab,debHan,walMel -b  sacCer,schPom 
    
    #Generate the datamatrix
    for i in `ls *_*.*.csv | grep -vP "(GO|interproscan)" | cut -f 2 -d
    "." | sort -u`; do perl ./matrixPlotComparativeAnnotation.pl -l
    phiSp_phiScl.$i.csv,phiSp_aspRub.$i.csv,phiSp_penChr.$i.csv,phiSp_horWer.$i.csv,phiSp_canPar.$i.csv,phiSp_debFab.$i.csv,phiSp_debHan.$i.csv,phiSp_sacCer.$i.csv,phiSp_schPom.$i.csv,phiSp_walIch.$i.csv,phiSp_walMel.$i.csv,phiSp_phiSclpenChrcanPardebFabdebHanwalMel.$i.csv,phiSp_sacCerschPom.$i.csv,phiSp_aspRubhorWerwalIch.$i.csv,phiSclpenChrcanPardebFabdebHanwalMel_sacCerschPom.$i.csv,phiSpaspRubhorWerwalIch_sacCerschPom.$i.csv,phiSpaspRubhorWerwalIch_phiSclpenChrcanPardebFabdebHanwalMel.$i.csv
    -d 1 -t 1e-5 > dataMatrix.$i.csv; done 
    
    #For interproscan
    for i in `ls *_*.interproscan.csv | cut -f 2 -d "." | sort -u`; do
    perl ./matrixPlotComparativeAnnotation.pl -l
    phiSp_phiScl.$i.csv,phiSp_aspRub.$i.csv,phiSp_penChr.$i.csv,phiSp_horWer.$i.csv,phiSp_canPar.$i.csv,phiSp_debFab.$i.csv,phiSp_debHan.$i.csv,phiSp_sacCer.$i.csv,phiSp_schPom.$i.csv,phiSp_walIch.$i.csv,phiSp_walMel.$i.csv,phiSp_phiSclpenChrcanPardebFabdebHanwalMel.$i.csv,phiSp_sacCerschPom.$i.csv,phiSp_aspRubhorWerwalIch.$i.csv,phiSclpenChrcanPardebFabdebHanwalMel_sacCerschPom.$i.csv,phiSpaspRubhorWerwalIch_sacCerschPom.$i.csv,phiSpaspRubhorWerwalIch_phiSclpenChrcanPardebFabdebHanwalMel.$i.csv
    -d 1 -t 1e-5 -p /media/work/share/database/InterproscanDictionary.dat > dataMatrix.$i.csv; done 

    #For GO
    for i in `ls *_*.GO.csv | cut -f 2 -d "." | sort -u`; do  perl
    ./matrixPlotComparativeAnnotation.pl -l
    phiSp_phiScl.$i.csv,phiSp_aspRub.$i.csv,phiSp_penChr.$i.csv,phiSp_horWer.$i.csv,phiSp_canPar.$i.csv,phiSp_debFab.$i.csv,phiSp_debHan.$i.csv,phiSp_sacCer.$i.csv,phiSp_schPom.$i.csv,phiSp_walIch.$i.csv,phiSp_walMel.$i.csv,phiSp_phiSclpenChrcanPardebFabdebHanwalMel.$i.csv,phiSp_sacCerschPom.$i.csv,phiSp_aspRubhorWerwalIch.$i.csv,phiSclpenChrcanPardebFabdebHanwalMel_sacCerschPom.$i.csv,phiSpaspRubhorWerwalIch_sacCerschPom.$i.csv,phiSpaspRubhorWerwalIch_phiSclpenChrcanPardebFabdebHanwalMel.$i.csv
    -d 1 -t 1e-5 > dataMatri x.$i.csv; done

    cut -f 1 dataMatrix.GO.csv | sed -r 's/"//g' > GOList
    Rscript ./clusterGOTerms.R -f GOList -o phiSpGOListClustered

    for i in `ls *_*.GO.csv | cut -f 2 -d "." | sort -u`; do  perl
    ./matrixPlotComparativeAnnotation.pl -l
    phiSp_phiScl.$i.csv,phiSp_aspRub.$i.csv,phiSp_penChr.$i.csv,phiSp_horWer.$i.csv,phiSp_canPar.$i.csv,phiSp_debFab.$i.csv,phiSp_debHan.$i.csv,phiSp_sacCer.$i.csv,phiSp_schPom.$i.csv,phiSp_walIch.$i.csv,phiSp_walMel.$i.csv,phiSp_phiSclpenChrcanPardebFabdebHanwalMel.$i.csv,phiSp_sacCerschPom.$i.csv,phiSp_aspRubhorWerwalIch.$i.csv,phiSclpenChrcanPardebFabdebHanwalMel_sacCerschPom.$i.csv,phiSpaspRubhorWerwalIch_sacCerschPom.$i.csv,phiSpaspRubhorWerwalIch_phiSclpenChrcanPardebFabdebHanwalMel.$i.csv
    -d 1 -t 1e-5 -p ./phiSpGOListClustered > dataMatrix.$i.csv; done 
    
    #Generate R plots
    for i in dataMatrix*; do Rscript ./plotMatrixEnrichment.R -f $i -o $i.svg; done
  }
  \label{code:countTotalParalog}
  \caption{shell command used to compute {\claImm} paralog and unique paralog.}
\end{listing}
\fi


%AA comparison
\ifcode
\begin{listing}
  \mint[breaklines,bgcolor=black,formatcom=\color{white},fontsize=\footnotesize]{shell}{
    
  \label{code:AAComparison}
  \caption{Command used to generate the rooted tree}
\end{listing}
\fi

\TODO{AA analysis with correlation analysis, apply correlation
  analysis to eggnog annotation and IPR}
%%%%%%%%%%%%%%%%%%%%%%%%%%%%%%%%%%%%%%%%%%%%%%%%%%%%%%%%%%%%%%%%%%%%%%%%%%%%%%%%

%CAFE
%%%%%%%%%%%%%%%%%%%%%%%%%%%%%%%%%%%%%%%%%%%%%%%%%%%%%%%%%%%%%%%%%%%%%%%%%%%%%%%%
CAFE was used to estimate the number of families that underwent
significant contraction and expansion along the
phylogenetic tree derived from the multiple sequence alignment.
The families were taken from the Cazy, COG, GO, IPR, Merops, Pfam and
tcdb annotations. Of all 12 genomes, {\phiSp} has the largest number of family
contractions in each annotation categories. On the other hand
{\phiScl}, {\penChr} and {\horWer} exhibit the highest number of
expansions. Still in the case of {\horWer} the expansions are probably due
to the recent genome duplication. 

The families of rapidly evolving members in {\phiSp} were studied. GO
terms related to amino acid transmembrane transport (GO:000333,-9),
response to oxidative stress  (GO:0006979, -4) and nucleoside
metabolism (GO:0009116, -6) were shrinking. Interestingly amino acid
transmembrane transport is also significantly shrinking in the two
other halophiles, but not in the other genomes. 

The analysis of TCDB families shows that the
Acid-Polyamine-Organocation (APC) (2.A.3) Family is shrinking in
{\phiSp} and {\aspRub} by 12 and 11 genes, further indicating that the
number of amino acid transporters are decreasing in
halophiles. Interestingly, the Ssy5 peptidase  family (S64) grew by
two members in {\phiSp}. In {\sacCer} this protein is part of a plasma
membrane sensor able to detect the presence of extracellular amino
acid and triggers the signalling pathway leading to induction of an
amino acid permease~\cite{abdel2004}. The iron/lead transporter
(2.A.108 -7), the transient receptor potential Ca$^{2+}$ channel
family (1.A.4 -6), the oligopeptide transporter family (2.A.67 -5) and
aquaporins (1.A.8 -3) are all shrinking in {\phiSp}. Tripartite
ATP-independent periplasmic transporter grew by 7 members, i.e. TRAP-T
(2.A.56, IPR010656) in {\phiSp}. In \textit{Halomonas   elongata},
TRAP-T  is responsible for the uptake of the compatible solutes
ectoine and hydroxyectoine~\cite{Grammann2002}. {\phiScl} shows an
increase in 2.A.3 (+22) and 3.A.9 (+10) and a strong decrease in H+ or
Na+-translocating NADH Dehydrogenase Family 3.D.1 (-14). A similar
analysis in {\walIch} indicates that the interproscan family with the
largest increase is IPR001338, i.e. Hydrophobins a major contributor
to the {\walIch} halophilism. 


\ifcode
\begin{listing}
  \mint[breaklines,bgcolor=black,formatcom=\color{white},fontsize=\footnotesize]{shell}{
    for i in aspRub canPar debFab debHan horWer penChr phiSp phiScl
    sacCer schPom walIch walMel; do printf $i"#####\n"; cat
    tcdbCAFE.csv.small.results.cafe.report_fams.txt | grep $i | grep
    -Po "[A-Z0-9:]+\[(\+|-)\d+\*\]" | sed -r 's/([\[|\]|\*])/ /g' |
    sort -k 2,2gr  | perl -lane 'if(abs($F[1])>0) {print;}' | perl
    -lane 'print; printf `grep  $F[0]
    /media/work/share/database/InterProscanDictionary.dat`' | paste -
    - | sort -k 2,2; done
\end{listing}
\fi

There are 19 rapidly evolving COG families in {\phiSp}. Two growing
families 0PYD1 (+7) and 0PRTC (+3) have no known functions. Other
growing families were related to Methyltransferase 0PUVA (+5), 0PG34
Cytochrome P450 (+3), 0PFE6 MFS-multidrug transporter (+3), 0PP11 Ribose
5-phosphate isomerase (+2) and secretory lipase 0PG8W (+2). The MFS
toxin-efflux-pump 0QE2U (-2), carobydrate transporter 0PFF2 (-2) and oligopeptide
transporter 0PFPV (-3) are contracting. 

Analysis of the depleted interproscan families show that Major
facilitator superfamily (IPR011701 -85), Zn(2)-C6 fungal-type
DNA-binding domain (IPR001138 -76), the NAD(P)-binding\_domain
IPR016040 (-67), Transcription factor domain IPR007219 (-53), Protein
kinases domain IPR011009 (-51), Cytochrome P450 IPR001128 (-16) are
under pressure. Members of IPR011009 are also strongly depleted in
{\aspRub} (-50) and to a lesser extent in {\canPar} (-16) and
{\debFab} (-13). On the other hand two families were growing during
{\phiSp} evolution: IPR000254 (+6), the fungal cellulose binding
domain family and the TRAP transport system (IPR010656, +7), which is
responsible for uptake of compatible solutes. 

Carbohydrate-active enzyme in {\phiSp} are shrinking. Three auxiliary
enzymes (AA) family  AA1 (multi-copper oxidase), AA2 (lignin modifying
oxidase) and AA3 (glucose-methanol-choline peroxidase) lose 20, 7 and 5 members
respectively. Similarly {\aspRub} lost 6 members of the AA2
family. Glycoside hydrolase (GH) families GH28 (-12), GH43(-7), GH0
(-6), GH78 (-21) and Carbohydrate esterase (CE) CE0(-6) and CE14
(-11). 


\ifcode
\begin{listing}
  \mint[breaklines,bgcolor=black,formatcom=\color{white},fontsize=\footnotesize]{shell}{
    #PREPARE TREE
    library("ape")
    library("phytools")
    phy<-read.tree(file="../old/oneToOneHomolog.phy.contree")
    #Put the midpoint in the middle of the tree.
    phy2<-ladderize(midpoint.root(phy))
    write.tree(phy2,file="./rootedTree.tree")
    #add time + ultrametric
    python ~/bin/CAFE/python_scripts/cafetutorial_prep_r8s.py -i ./rootedTree.tree  -o test  -s 61283 -p '(S.cerevisiae,W.Mellicola)' -c '621'  
    ~/bin/r8s -b -f test > test2.txt
    tail -n 1 test2.txt | cut -f 16- > test_ultrametric.txt
    #In case iaembe appears, remove it
    #PREPARE DATA. 
    
    
    #COG DATA
    
    for i in *emapper.annotations; do cut -f 10 $i | grep -v "#" | cut -f 1 -d "|" | sort | uniq -c > CAFE/$i.count; done

    cut -f 10 *.annotations | cut -f 1 -d  "|" | grep -v "#" | sort -u > CAFE/listOFCOG 
    
    cat listOFCOG | xargs -I {}  sh -c 'printf {}"\t";grep -c {} *.annotations.count' | paste - - - - - - - - - - - - | sed -r 's/[a-z][^:]+//g' | sed -r 's/://g' > COG_data.tab 
    mv COG_data.tab COGCAFE.csv
    #OTHER DATA
    perl ./convertFAToCafe.pl -d .. -s aspRub,canPar,debFab,debHan,horWer,penChr,phiScl,phiSp,sacCer,schPom,walIch,walMel
    #For each 
    #filter family with more than 100 elements 
    for i in *CAFE.csv; do cat $i | perl -lane 'my $max = (sort {$b <=> $a} @F[2..$#F])[0];if($max>100 || $F[1]=~/FAMILY/){print;}' > $i.large;done
    for i in *CAFE.csv; do cat $i | perl -lane 'my $max = (sort {$b <=> $a} @F[2..$#F])[0];if($max<100 || $F[1]=~/FAMILY/){print;}' > $i.small;done
    
    #Train on sm1all and apply on big
    for i in {Cazy,Merops,interproscan,GO,tcdb,Pfam,COG}CAFE.csv.small; do
    echo "load -i $i -t 8 -l $i.log" > $i.cf
    echo "tree ((((((phiScl:60.617324,phiSp:60.617324):40.456106,aspRub:101.073430):28.880877,penChr:129.954307):171.714012,horWer:301.668319):270.062776,(((debFab:23.857217,debHan:23.857217):146.863667,canPar:170.720885):158.567076,sacCer:329.287961):242.443134):49.268905,((walMel:82.636765,walIch:82.636765):510.880936,schPom:593.517701):27.482299)" >> $i.cf
    echo "lambda -s -t  ((((((1,1)1,1)1,1)1,2)1,(((1,1)1,1)1,1)1)1,((1,1)1,1)1)" >> $i.cf
    echo "report $i.results" >> $i.cf
    done
    

    for i in {Cazy,Merops,interproscan,GO,tcdb,Pfam,COG}CAFE.csv.large; do
    echo "load -i $i -t 8 -l $i.log" > $i.cf
    echo "tree ((((((phiScl:60.617324,phiSp:60.617324):40.456106,aspRub:101.073430):28.880877,penChr:129.954307):171.714012,horWer:301.668319):270.062776,(((debFab:23.857217,debHan:23.857217):146.863667,canPar:170.720885):158.567076,sacCer:329.287961):242.443134):49.268905,((walMel:82.636765,walIch:82.636765):510.880936,schPom:593.517701):27.482299)" >> $i.cf
    printf  "lambda -t ((((((1,1)1,1)1,1)1,1)1,(((1,1)1,1)1,1)1)1,((1,1)1,1)1) -l " >> $i.cf
    j=`echo $i | sed -r 's/large/small/g'`
    printf $j
    tail $j.log | grep -P "^Lambda" | cut -f 3 -d " " | tail -n 1 | sed -r 's/,/ /g' >> $i.cf
    echo "report $i.results" >> $i.cf
    done
    #results summary
    python ~/bin/CAFE/python_scripts/cafetutorial_report_analysis.py -i smallCOG.multipleLambda.cafe -o smallCOG.multipleLambda.summary
    #Plot results in R
  \label{code:CAFE}
  \caption{Command used to generate the rooted tree}
\end{listing}
\fi




\ifcode
\begin{listing}
  \mint[breaklines,bgcolor=black,formatcom=\color{white},fontsize=\footnotesize]{shell}{
    perl ./getOverUnderFuncAnno2.pl -d . -s phiSp -b sacCer,schPom
    perl ../summarizeFuncAnno.pl -C Light_Blue,Light_Green -p  /media/work/share/database/materialPalette.csv~  -d . -s "phiSp_sacCerschPom.(Pfam|Merops|Cazy)" -F 1  -P 1e-3 -S 0.2 >
    overRepresented && fingerPrint.R -f overRepresented -c 1
    perl ../summarizeFuncAnno.pl -C Light_Blue,Light_Green -p  /media/work/share/database/materialPalette.csv~  -d . -s "phiSp_phiScl.(Pfam|Merops|Cazy)" -F 0  -P 1e-3 -S 0.2 >
    underRepresented && fingerPrint.R -f underRepresented -c 1 
  }
  \label{code:enrichedTerm}
  \caption{Command used to generate the enriched terms for each
    annotation catergory and barplot for some select categories}
\end{listing}
\fi

%MATRIX PLOT
\ifcode
\begin{listing}
  \mint[breaklines,bgcolor=black,formatcom=\color{white},fontsize=\footnotesize]{shell}{
    parallel -j 8 "perl ./getOverUnderFuncAnno2.pl -d . -s {1} -b {2}" ::: phiSp ::: phiScl aspRub penChr horWer canPar debFab debHan sacCer schPom walIch walMel
    #We only look at IPR at the beginning
    
  }
  \label{code:matrixPlot}
  \caption{Command used to generate the enriched terms for each
    annotation catergory and barplot for some select categories}
\end{listing}
\fi





\subsection{Horizontal Gene Transfer}
HGTFinder~\cite{Nguyen2015} was used to look for horizontally transferred
bacterial genes. 76 genes in {\phiSp} were found to be horizontally transfered
with a q-value of transfer $<0.05$ (See Supplementary Table 1). Among
them 65 are partial proteins and 11 are complete genes. Among those
11 genes, 4 have protein domains not found in any other genes in
{\phiSp}: PHISP\_08405 has a Ribbon-helix-helix domain, PHISP\_08807
is a gluconate transporter, PHISP\_08812 an urea transporter and
PHISP\_08441 is a transposase. 

Functional enrichment analysis of all horizontally transferred genes
indicates that genes related to cation efflux (IPR027469\E{4.1}{3},
tcdb:2A4\E{1.8}{3}), ABC transporter (tcdb:3A1 \E{2.0}{5}), Transposase
(IPR004291 0.012, IPR002514 0.012), TRAP-C4-dicarboxylate transport
(IPR010656 0.028) and nucleoside triphosphate hydrolase (IPR27417 0.028)
(See Table~\ref{tab:HGTEnrichment}). Genes related to The tellurium
ion resistance   (2.A.109, PHISP\_08523), malonate symporter
(2.A.70, PHISP\_08837),   monovalent cation/proton antiporter family
(2.A.63, PHISP\_08638) are only found among the set of horizontally
transferred genes. Gene ontology relative to DNA recombination and ion
transfer are overrepresented but not significantly (0.05 $<$ q-value $<$
0.1) (See Supplementary Table S1).

We look at the distance tree of 100 closest Blastp results against nr
of three horizontally transferred transporters. Homologues of
PHISP\_08463 and PHISP\_08184, a  proton pump and a cation efflux
protein are mainly located in the class of $\alpha$-proteobacteria,
while PHISP\_08633, an ABC transporter, has homologues mainly in the
class of green algae Mamiellophycae (Figure~\ref{fig:treeHGT}).



\begin{figure}[!h]
  \centering
  \includegraphics[width=\linewidth]{./treeHGT.pdf}
  \caption{\label{fig:treeHGT} BLA
  } 
\end{figure}


\ifcode
\begin{listing}
  \mint[breaklines,bgcolor=black,formatcom=\color{white},fontsize=\footnotesize]{shell}{
    #selfblast
     blastp -db ./phiSp.faa -query phiSp.faa -evalue 1e-3 -outfmt 6
     #blast against nr
     for i in x??; do echo ``blastp -db nr -query $i -evalue 1e-3 -outfmt 6  -num_threads 16 > $i.blast.nr `` | qsub -V -cwd; done
     #retain all genes that are in nr.tax
     cat ~/bin/HGTFinder/nr.tax | cut -f 2 -d ``|'' | fgrep -wf - phiSpVsNr.res  > phiSpVsNrOverlapp.res
    #Apply HGTFinder
     ./HGTFinder -d /global/lv70539/htafer/horizontalGeneTransfer/phiSpVsNrOverlapp.res -s /global/lv70539/htafer/horizontalGeneTransfer/phiSp.self.blast     -t 569365  -r 0.2,0.3,0.4,0.5,0.6,0.7,0.8,0.9 -o test > log
     #We get phiSp putativeHGT
  }
  \label{code:HGTFinder predictions}
  \caption{HGTFinder commands}
\end{listing}
\fi

\ifcode
\begin{listing}
  \mint[breaklines,bgcolor=black,formatcom=\color{white},fontsize=\footnotesize]{shell}{
    #Check which transfererd gens is complete
    cat putativeHGT | fgrep -wf -  ../../phiSp.oneLine.faa  -A 1 | grep -P "^M.+\*$" -B 1 | grep -Po  "PHISP\_\d+" | sort -u > putativeHGT.complete
    #Compute enrichment for the set of complete sequences putatively transferred
    parallel -j 8 'j=`echo {} | sed -r "s/phiSp.ann.//"`; enrichmentStat.R -b {} -d putativeHGT.complete -t $j -D /media/htafer/work/share/database/InterProscanDictionary.dat;' ::: phiSp.ann.*
  }
  \label{code:HGTFinder analysis}
  \caption{HGTFinder commands}
\end{listing}
\fi

\ifcode
\begin{listing}
  \mint[breaklines,bgcolor=black,formatcom=\color{white},fontsize=\footnotesize]{shell}{
    #Blast Get results from ncb
    curl -f
    "https://blast.ncbi.nlm.nih.gov/Blast.cgi?RESULTS_FILE=on&RID=UDYDGJWN016&FORMAT_TYPE=CSV&FORMAT_OBJECT=Alignment&DESCRIPTIONS=100&ALIGNMENT_VIEW=Tabular&CMD=Get"
    > blastHitReport.csv
    cat blastHitReport.csv  | cut -f 2 -d "," | cut -f 4 -d "|" | sort
    -u > ids
    #Taxonomy
    cat ids | split -l 100 | parallel -j 20 "perl ~/bin/testEUtils.pl  < {} > {}.res" ::: x??
    #
    parallel -j 20 "perl ~/bin/testEUtils.pl < {} > {}.res" ::: x??
    #
    for i in `cat ids`; do grep -w $i  seqTax.csv | head -n1 ; done >
    seqTax.csv.final
    cut -f 1 blastHitReport.csv -d , | cut -f 2 -d "." | sort -u | fgrep -wf - ../../../phiSp/phiSpSingleLine.faa  -A 1 | grep -v "\--" | cut -f 1 -d " " | perl -lane 'if($F[0]=~/^>/){printf; printf " Eurotiomycetes Aspergillaceae Eurotiales  Ascomycota "}else{print $F[0]}' | cut -f 2- -d "." >> seqTax.csv.final
    #Get Sequence for each alignment
    for i in `cut -f 1 blastHitReport.csv -d "," | cut -f 2 -d "." | sort -u`; do grep $i blastHitReport.csv | cut -f 2 -d "," | cut -f 4 -d "|"  | sort -u | fgrep -wf - seqTax.csv.final | sed -r 's/\*//g' | perl -lane 'printf ">$F[0]\n$F[5]\n"' > $i.temp && grep $i seqTax.csv.final | sed -r 's/\*//g' | perl -lane 'printf ">$F[0]\n$F[5]\n"' >> $i.temp ; done
    #Generate Alignment
    parallel -j 8 " muscle -in {} -out  {}.phy" ::: *.temp
    #Generate Tree
    for i in *.temp.phy ; do FastTreeMP $i > $i.tree; done
    #Paint Tree
    Rscript ./colorTree.R -t seqTax.csv.final -T ./PHISP\_o08463.temp.phy.tree -d class

  }
  \label{code:HGTFinder Tree plots}
  \caption{HGTFinder commands}
\end{listing}
\fi

\subsection{Amino Acid bias}

\ifcode
\begin{listing}
  \mint[breaklines,bgcolor=black,formatcom=\color{white},fontsize=\footnotesize]{shell}{
    #Generate Data
    ################################################################################
    #Amino acid protein ortho analysis
    ################################################################################
    
    #For each gene in genelist, extract sequence count aa and print results.
    for i in `seq 1 12`; do species=`head -n $i speciesList.csv  | tail -n 1`; export SPECIES=$species; cut -f 4- geneList.csv | cut -f $i | cut -f 1 -d "," | xargs -I {} sh -c 'grep -A 1 {} allClean.FAA | grep -v ">" > ./string ; length=`grep -Po [A-Z] string | wc -l`; export LENGTH=$length;  for k in `cat AAlist`; do  printf $k"\t"$SPECIES"\t"$LENGTH"\t"; grep $k -o ./string | grep -c $k   ; done  '  ; done > aaCountPerGene.csv
    #Now process it into R
    
    ################################################################################
    #Amino acid secreted proteins
    ################################################################################
    
    for i in *.SPRES; do cat $i | grep -P "\sY\s" | grep -v "\-TM" | cut -f 1 -d " " | cut -f 2 -d "." > $i.secreted.id; done
    
    
    for i in *.id; do export SPECIES=`echo $i | cut -f 1 -d "."`; cat $i | xargs -I {} sh -c 'grep -A 1 {} allClean.FAA | grep -v ">" > ./string ; length=`grep -Po [A-Z] string | wc -l`; export LENGTH=$length; for k in `cat AAlist`; do printf $k"\t"$SPECIES"\t"$LENGTH"\t";   grep $k -o ./string | grep -c $k   ; done  '  ; done >> aaCountPerGeneSecreted.csvaaC	


    #Analysis Data
    ################################################################################
	#Proteinortho AA analysis
	
	library(ggplot2)
	library(ggsci)
	library(grid)
	library(gridExtra)
	
	# Multiple plot function
	#
	# ggplot objects can be passed in ..., or to plotlist (as a list of ggplot objects)
	# - cols:   Number of columns in layout
	# - layout: A matrix specifying the layout. If present, 'cols' is ignored.
	#
	# If the layout is something like matrix(c(1,2,3,3), nrow=2, byrow=TRUE),
	# then plot 1 will go in the upper left, 2 will go in the upper right, and
	# 3 will go all the way across the bottom.
	#
	multiplot <- function(..., plotlist=NULL, file, cols=1, layout=NULL) {
	    library(grid)
	    
	                                        # Make a list from the ... arguments and plotlist
	    plots <- c(list(...), plotlist)
	    
	    numPlots = length(plots)
	    
	                                        # If layout is NULL, then use 'cols' to determine layout
	    if (is.null(layout)) {
	                                        # Make the panel
	                                        # ncol: Number of columns of plots
	                                        # nrow: Number of rows needed, calculated from # of cols
	        layout <- matrix(seq(1, cols * ceiling(numPlots/cols)),
	                         ncol = cols, nrow = ceiling(numPlots/cols))
	    }
	    
	    if (numPlots==1) {
	        print(plots[[1]])
	        
	    } else {
	                                        # Set up the page
	        grid.newpage()
	        pushViewport(viewport(layout = grid.layout(nrow(layout), ncol(layout))))
	        
	                                        # Make each plot, in the correct location
	        for (i in 1:numPlots) {
	                                        # Get the i,j matrix positions of the regions that contain this subplot
	            matchidx <- as.data.frame(which(layout == i, arr.ind = TRUE))
	            
	    #        print(plots[[i]], vp = viewport(layout.pos.row = matchidx$row,
	    #                                        layout.pos.col = matchidx$col))
	        }
	    }
	}
	
	
	
	
	
	
	
	dataProteinOrtho<-read.table("./aaCountPerGene.csv", header=T)
	dataProteinOrtho$CC="ProteinOrtho"
	dataProteinOrtho$prop<-dataProteinOrtho$Count/dataProteinOrtho$gL
	dataProteinOrtho$type<-factor(dataProteinOrtho$type,levels=c("ct","ht","hp"),ordered=T)
	dataSecreted<-read.table("./aaCountPerGeneSecreted.csv", header=T)
	dataSecreted$CC="Secreted"
	dataSecreted$prop<-dataSecreted$Count/dataSecreted$gL
	dataSecreted$type<-factor(dataSecreted$type,levels=c("ct","ht","hp"),ordered=T)
	
	#merge The data
	data<-rbind(dataProteinOrtho,dataSecreted)
	#data$prop<-data$Count/data$gL
	data$type<-factor(data$type,levels=c("ct","ht","hp"),ordered=T)
	
	ggplot(data, aes(AA, prop))+geom_boxplot(aes(colour=type))+facet_wrap(~CC)
	
	aa=NULL
	cthtpv=NULL
	hthppv=NULL
	cthppv=NULL
	
	cthtks=NULL
	hthpks=NULL
	cthpks=NULL
	wtest=NULL
	CC=NULL
	for (i in as.vector(unique(data["AA"])[,1])){
	    aa=append(aa,i)
	
	    temp=ks.test(data$prop[data[["AA"]]==i & data$type=="ct"& data$CC=="ProteinOrtho"], data$prop[data$AA==i & data$type=="ht"& data$CC=="ProteinOrtho"])
	    cthtpv=append(cthtpv,temp$p.value)
	    cthtks=append(cthtks,temp$statistic)
	    
	    temp=ks.test(data$prop[data[["AA"]]==i & data$type=="ht"& data$CC=="ProteinOrtho"], data$prop[data$AA==i & data$type=="hp"& data$CC=="ProteinOrtho"])
	    hthppv=append(hthppv,temp$p.value)
	    hthpks=append(hthpks,temp$statistic)
	
	    temp=ks.test(data$prop[data[["AA"]]==i & data$type=="ct"& data$CC=="ProteinOrtho"], data$prop[data$AA==i & data$type=="hp"& data$CC=="ProteinOrtho"])
	    cthppv=append(cthppv,temp$p.value)
	    cthpks=append(cthpks,temp$statistic)
	
	
	
	    CC=append(CC,"ProteinOrtho")
	    aa=append(aa,i)
	    temp=ks.test(data$prop[data[["AA"]]==i & data$type=="ct"& data$CC=="Secreted"], data$prop[data$AA==i & data$type=="ht"& data$CC=="Secreted"])
	    cthtpv=append(cthtpv,temp$p.value)
	    cthtks=append(cthtks,temp$statistic)
	    
	    temp=ks.test(data$prop[data[["AA"]]==i & data$type=="ht"& data$CC=="Secreted"], data$prop[data$AA==i & data$type=="hp"& data$CC=="Secreted"])
	    hthppv=append(hthppv,temp$p.value)
	    hthpks=append(hthpks,temp$statistic)
	    
	    temp=ks.test(data$prop[data[["AA"]]==i & data$type=="ct"& data$CC=="Secreted"], data$prop[data$AA==i & data$type=="hp"& data$CC=="Secreted"])
	    cthppv=append(cthppv,temp$p.value)
	    cthpks=append(cthpks,temp$statistic)
	    CC=append(CC,"Secreted")
	}
	#wtest=rbind(wtest,data.frame(i,ctht,hthp,cthp,CC))
	
	#Here we store the statistics
	wtest = data.frame(aa,cthtpv,hthppv,cthppv,cthtks,hthpks,cthpks,CC)
	
	#interesting <- wtest[wtest$ctht > wtest$cthp & wtest[,2] < quantile(wtest[,2])[3] &  wtest[,3] < quantile(wtest[,3])[3] &  wtest[,4] < quantile(wtest[,4])[3], ]
	#interesting<-wtest[wtest$cthtpv  >= wtest$cthppv,]
	
	#Plot boxplot based on selected statistics
	interestingSecreted<-wtest[wtest$CC=="Secreted",]
	interestingProteinOrtho<-wtest[wtest$CC=="ProteinOrtho",]
	
	interestingSecreted<-tail(interestingSecreted[order(interestingSecreted$cthpks),],5)
	interestingProteinOrtho<-tail(interestingProteinOrtho[order(interestingProteinOrtho$cthpks),],5)
	
	interesting<-rbind(interestingSecreted,interestingProteinOrtho)
	
	
	
	a<-ggplot(dataProteinOrtho[dataProteinOrtho[["AA"]] %in% interestingProteinOrtho$aa, ] , aes(AA, prop))+geom_boxplot(aes(fill=type)) + scale_fill_npg()#+facet_wrap(~type+CC+AA,scale="free")
	b<-ggplot(dataSecreted[dataSecreted[["AA"]] %in% interestingSecreted$aa, ] , aes(AA, prop))+geom_boxplot(aes(fill=type)) + scale_fill_npg()#+facet_wrap(~type+CC+AA,scale="free") 
	multiplot(a,b,cols=2)
	
	
	
	
	wtestSecreted<-wtest[wtest$CC=="Secreted",]
	wtestSecreted<-#scale(wtestSecreted[c(-1,-8)],center=TRUE,scale=TRUE)
	wtestSecreted<-as.data.frame(wtestSecreted)
	wtestSecreted$CC=wtest$CC[wtest$CC=="Secreted"]
	wtestSecreted$aa=wtest$aa[wtest$CC=="Secreted"]
	
	wtestProteinOrtho<-wtest[wtest$CC=="ProteinOrtho",]
	#wtestProteinOrtho<-scale(wtestProteinOrtho[c(-1,-8)],center=TRUE,scale=TRUE)
	wtestProteinOrtho<-as.data.frame(wtestProteinOrtho)
	wtestProteinOrtho$CC=wtest$CC[wtest$CC=="ProteinOrtho"]
	wtestProteinOrtho$aa=wtest$aa[wtest$CC=="ProteinOrtho"]
	rbind(wtestSecreted,wtestProteinOrtho)
	
	wtestPlot<-NULL
	wtestPlot$aa=wtest$aa[wtest$CC=="ProteinOrtho"]
	wtestPlot$cthpPO=wtestProteinOrtho$cthpks
	wtestPlot$cthpSE=wtestSecreted$cthpks
	wtestPlot<-as.data.frame(wtestPlot)
	
	
	wtestPlot$Family<-c(
	 "Hydrophobic",
	 "Other      ",
	 "Negative   ",
	 "Negative   ",
	 "Hydrophobic",
	 "Other      ",
	 "Positive   ",
	 "Hydrophobic",
	 "Positive   ",
	 "Hydrophobic",
	 "Hydrophobic",
	 "Polar      ",
	 "Other      ",
	 "Polar      ",
	 "Positive   ",
	 "Polar      ",
	 "Polar      ",
	 "Hydrophobic",
	 "Hydrophobic",
	 "Hydrophobic")
	
	wtestPlot$Class<-c(
	"Aliphatic",
	"Sulfur   ",
	"Acidic   ",
	"Acidic   ",
	"Aromatic ",
	"Aliphatic",
	"Basic    ",
	"Aliphatic",
	"Basic    ",
	"Aliphatic",
	"Sulfur   ",
	"Amidic   ",
	"Aliphatic",
	"Amidic   ",
	"Basic    ",
	"Hydroxyli",
	"Hydroxyli",
	"Aliphatic",
	"Aromatic ",
	"Aromatic ")    
	
	colorManual<-c("#fb61d7","#00b6eb","#00c094","#53b400","#c49a00","#f8766d","#a58aff")
	
	a<-ggplot(dataProteinOrtho[dataProteinOrtho[["AA"]] %in% interestingProteinOrtho$aa, ] , aes(AA, prop))+geom_boxplot(aes(fill=type)) + scale_fill_manual(values=colorManual)#+facet_wrap(~type+CC+AA,scale="free")
	b<-ggplot(dataSecreted[dataSecreted[["AA"]] %in% interestingSecreted$aa, ] , aes(AA, prop))+geom_boxplot(aes(fill=type)) + scale_fill_manual(values=colorManual)#+facet_wrap(~type+CC+AA,scale="free")
	
	scatter<-ggplot(wtestPlot, aes(cthpPO,cthpSE,color=Class))+geom_point(aes(shape = Family,stroke=2), size = 7, show.legend = TRUE) + scale_color_manual(values=colorManual)+scale_shape_manual(values=c(0,1,2,5,6)) +geom_text(aes(label=aa))
	
	q<-plot_grid(plot_grid(a,b,nrow=2),scatter,ncol=2)
  }
  \label{code:Amino Acid Composition Bias Analysis}
  \caption{HGTFinder commands}
\end{listing}
\fi


\begin{figure}[!h]
  \centering
  \includegraphics[width=\linewidth]{./finalAA_Analysis.pdf}
  \caption{\label{fig:AABIAS} BLA  } 
\end{figure}



\subsection{Potential for Secondary Metabolite}

\ifcode
\begin{listing}
  \mint[breaklines,bgcolor=black,formatcom=\color{white},fontsize=\footnotesize]{shell}{
#Get annotation
parallel -j 8 "grep -w {} ../phiSp.PHISP.gff3 | smashAddProductToCDS.pl > {}.gff3" ::: `cut -f 1 ../phiSp.PHISP.gff3 | grep -v "#" | grep -vP "^$" | sort -u ` 

#Get sequence
parallel -j 8 "grep -w {} ../../phiSp.fasta -A 1 > {}.fa" ::: `cut -f 1 ../phiSp.PHISP.gff3 | grep -v "#" | grep -vP "^$" | sort -u `

parallel -j 8 "seqret -sequence {}.fa -feature -fformat gff -fopenfile {}.gff3 -osformat genbank -auto -osname {} -osextension gb " ::: `ls *.fa | sed -r 's/.fa//g'`
#Add product instead of note // Add it to CDS
parallel -j 8 "sed -i -r 's/product/protein_id/' {}" ::: *.gb

source ~/bin/as3/bin/activate
parallel -j 8 " python ~/bin/AS3/antismash/run_antismash.py -c 8 --logfile {}.log --input-type nucl --smcogs --inclusive --borderpredict --verbose --taxon fungi --clusterblast --subclusterblast --knownclusterblast --asf --all-orfs {} -v " ::: *.gb

#parsing of the data
for i in `find . -name \*.1.final.gbk`; do perl ~/bin/parseSMASHGBK.pl -g $i; done | sort -k 1,1 -k 2,2g

#for the other genomes we use ~/bin/launch.Antismash
for i in horWer penChr phiScl sacCer schPom walIch walMel; do j=$i.fa; k=$i.gff3; printf "launch.Antismash $k $j\n"; done

#!/bin/bash
GFF=$1
DIR=`echo $GFF | sed -r 's/.gff.+//'`
echo "GFF:$GFF DIR:$DIR"
#genome fasta file with 1sequence per line
RMDBFA=$2
echo "PREPROCESS GFF"
mkdir -p $DIR
parallel   -j 8 "grep -w {} $GFF > $DIR/{}.gff3" ::: `cut -f 1 $GFF | grep -v "#" | grep -vP "^$" | sort -u ` 
echo "PREPROCESS SEQUENCE"
#Get sequence
parallel  -j 8 "grep -w {} $RMDBFA -A 1 > $DIR/{}.fa" ::: `cut -f 1 $GFF | grep -v "#" | grep -vP "^$" | sort -u`

echo "CD $DIR"
cd $DIR
echo "CONVERT GFF TO GB"
parallel  -j 8 "gff_to_genbank.py {}.gff3 {}.fa" ::: `ls *.fa | sed -r 's/.fa//g'`

echo "START ANTISMASH"
source ~/bin/as3/bin/activate
parallel -j 3  "python ~/bin/AS3/antismash/run_antismash.py -c 8 --logfile {}.log --input-type nucl --smcogs --inclusive --borderpredict --verbose --taxon fungi --clusterblast --subclusterblast --knownclusterblast --asf --all-orfs {} -v " ::: *.gb
#Get stat for phiSp
for i in `find  ../../annotation/secondaryMetabolites/smash/ -name \*BGC.txt`; do printf $i"\n"; cat $i | cut -f 2 ; done | sort | uniq -c
#Get stat for the other species
Get coordinates from antismash
for i in schPom debHan debFab canPar walIch sacCer penChr horWer phiScl; do printf $i"\n"; find $i -name \*BGC.txt | xargs -I {} cat {} >> $i"SMASHBCG.txt"; done
find ../../annotation/secondaryMetabolites/smash/ -name  \*BGC.txt | xargs -I {} cat {} >> phiSp"SMASHBCG.txt" 
#Convert it to gff3
for i in *BCG.txt; do cat $i  | sed -r 's/_c[0-9]+//' | sed -r 's/ /_/g' | grep -v BGC | sed -r 's/\t0;/\t1;/' | sed -r 's/([0-9]);([0-9])/\1\t\2/' | perl -lane 'printf "$F[0]\tsmash\tsmash\t$F[3]\t$F[4]\t.\t+\t.\t$F[1]\n"' > $i.gff3; done
#SMURF
#Download smurf from mail attachment
#cp BackBone to xxxxxBBG.txt
#convert it to gff3
for i in *BBG.txt; do cat $i | perl -lane '($start,$end,$strand)=($F[4]<$F[5]?($F[4],$F[5],"+"):($F[5],$F[4],"-")); printf "$F[2]\tsmurf\tsmurf\t$start\t$end\t.\t$strand\t.\t$F[6]\n"' | tail -n +2 > $i.gff3; done

#Look at intersection
for i in aspRub canPar debFab debHan horWer penChr phiSp phiScl sacCer schPom walIch walMel; do bedtools intersect -wa -wb -a $i"BBG.txt.gff3" -b $i"SMASHBCG.txt.gff3"; bedtools intersect -v -a $i"BBG.txt.gff3" -b $i"SMASHBCG.txt.gff3";  bedtools intersect -v -b $i"BBG.txt.gff3" -a $i"SMASHBCG.txt.gff3"; done
   }
  \label{code:smash}
  \caption{Smash commands}
\end{listing}
\fi


\begin{table}
  \small
  \begin{tabular}{|p{1.5cm}|l|l|l|l|l|l|l|l|l|l|l|l|}
    \hline
    Secondary Metabolite Family& phiSp& phiScl&	aspRub& penChr&horWer&	 debFab& debHan& canPar&  sacCer&schPom&walIch&walMel\\
    \hline                                     	               	      	                  
    \hline                                     	               	      	                  
    DMAT&	                 1&      0&	 3&	1&     	 0&   	  0&	0&      0&        0&	0&	0&	0\\
    indole&	                 2&      0&	 4&	0&     	 1&   	  0&	0&      0&        0&	0&	0&	0\\
    hybrid&	                 1&      3&	 0&	2&     	 0&   	  0&	0&      0&        0&	0&	0&	0\\
    nrps&	                 23&     41&	 22&	28&    	 18&  	  0&	0&      0&        0&	1&	4&	3\\
    NRPS-Like&	                 7&      7&	 7&	13&    	 9&   	  0&	0&      0&        0&	0&	2&	1\\
    PKS&	                 9&      12&	 13&	23&    	 4&   	  1&	1&      1&        1&	0&	1&	1\\
    PKS-Like&	                 2&      1&	 2&	1&     	 4&   	  1&	1&      1&        1&	0&	0&	0\\
    cf\_fatty\_acid\_&	         4&      1&	 1&	1&     	 4&   	  2&	2&      2&        2&	2&	0&	0\\
    cf\_putative&	         36&     40&	 54&	15&    	 24&  	  17&	21&     17&       9&	8&	11&	9\\
    cf\_saccharide&	         0&      0&	 0&	0&     	 0&   	  1&	1&      1&        0&	0&	0&	0\\
    other&	                 8&      5&	 4&	1&     	 13&  	  1&	1&      1&        1&	2&	1&	1\\
    siderophore&	         0&      0&	 0&	1&     	 0&   	  0&	0&      0&        0&	0&	0&	0\\
    terpene&	                 3&      2&	 3&	0&     	 3&   	  1&	2&      1&        1&	1&	1&	1\\
    \hline                                     	               	      	                  
    sum    &                     96& 	 112&  	113&     86&   	  80& 	  24&     29&    24&       15&     14&     20&     16\\
    \hline
  \end{tabular}
  \caption{\label{tab:cluster} Secondary metabolite biosynthesis gene
    clusters found in the twelwe genomes under studies with SMURF and SMASH. 
    Overlapping predictions were resolved by keeping the SMURF predictions.}
\end{table}


{\phiSp} and {\phiScl} have a total of 96 and 112 annotated secondary
metabolites backbone gene clusters (SMBGCs), respectively. They show a
similar SMBGCs profile as {\aspRub}, which is a phylogenetic
relative. The main difference between both \textit{phialosimplex}
species is high number of nrps-related cluster in {\phiScl} compared
to {\phiSp}. DMAT (smurf) and indole (smash) SGMBCs are present in
{\phiSp} and {\aspRub} but absent in {\phiScl}.  


\subsection{Detailed Analysis}
\subsubsection{HOG Pathway}

The genome of {\phiSp} was searched for homologues  of the high-osmolarity
glycerol response signalling pathway from {\sacCer}~\cite{Bahn2007}, {\walIch}~\cite{Zajc2014} and
\textit{Aspergillus}~\cite{Castro2014}. 


\begin{table}
  \begin{tabular}{|l|l|l|l|l|l|l|}
    Species &{\aspRub}	        &        {\phiScl}     &      {\phiSp}        &   {\sacCer}     & {\walIch}      \\
    Sho1    &EYE93144,1e-147	&PHISCL_08295,1e-147	&PHISP_07557,1e-144	&YER118C,0.0	&KUR62632,4e-28  \\
    Sln1    &EYE94687,2e-18*	&PHISCL_09956,6e-15*	&PHISP_05099,3e-15*	&YIL147C,0.0	&EOR01469,2e-18* \\
    Cdc42   &EYE93902,4e-118	&PHISCL_01735,6e-86	&PHISP_08179,7e-119	&YLR229C,2e-141	&EOR00654,1e-123 \\
    Cdc24   &EYE92189,1e-39	&PHISCL_09747,6e-39	&PHISP_06763,3e-38	&YAL041W,0.0	&EOR03906,1e-40  \\
    Ypd1    &EYE99886,3e-17	&PHISCL_09916,2e-21	&PHISP_03488,2e-20	&YDL235C,3e-89	&EOQ99342,2e-21  \\
    Ste20   &EYE90445,0.0	&PHISCL_02057,0.0	&PHISP_05030,0.0	&YHL007C,0.0	&EOQ99703,0.0    \\
    SSk1    &EYE93143,7e-60	&PHISCL_08297,1e-58	&PHISP_07559,8e-60	&YLR006C,0.0	&EOQ98783,3e-52  \\
    Ste50   &EYE96463,2e-20	&PHISCL_04915,3e-19	&PHISP_05595,2e-19	&YCL032W,0.0	&EOR03947,7e-13  \\
    Ste11   &EYE93420,5e-93	&PHISCL_09185,1e-95	&PHISP_00701,3e-93	&YLR362W,0.0	&EOQ99709,7e-92  \\
    SSk2    &EYE95333,2e-146	&PHISCL_01269,2e-147	&PHISP_03935,4e-147	&YCR073C,0.0	&EOR00620,3e-120 \\
    Pbs2    &EYE98904,3e-123	&PHISCL_06638,1e-122	&PHISP_07372,1e-67	&YJL128C,0.0	&EOR04233,1e-101 \\
    Hog1    &EYE92398,0.0	&PHISCL_09066,0.0	&PHISP_02745,0.0	&YLR113W,0.0	&EOQ98833,0.0    \\
    Skn7    &EYE91090,1e-46	&PHISCL_05564,4e-42	&PHISP_02412,2e-42	&YHR206W,0.0	&EOR04976,2e-46  \\ 
    \hline\hline
    PhkB    &EYE94256,0.0	&PHISCL_07220,0.0	&PHISP_00829,0.0	&YLR006C,2e-11*	&EOR01469,2e-45*  \\
    CrzA    &EYE99449,0.0	&PHISCL_03951,0.0	&PHISP_02476,0.0	&YNL027W,4e-52	&EOR00160,2e-26  \\ 
  \end{tabular}
  \caption{\label{tab:halophile} }
\end{table}

\subsubsection{Compatible Solutes}




\TODO{
MA 6005 Phialosimplex Sp. Aspergillus salisburgensis
MA      Phialosimplex Scl. Aspergilus sclerortialis
MA 6032 Aspergillus 
MA 6003 Aspergillus }

\section{Discussion}
%%In this study, for the first time, the transcriptome of a fungus shown
%%to grow on toluene~\cite{Blasi2016} was repoducibly sequenced. 
%%With the help of the transcriptomic data and a thorough comparative
%%genomics analysis, genes previoulsy reported to play a role in
%%metabolizing {\tol} in other fungi and bacteria~\cite{Parales2008}
%%were identified and their expression levels during the toluene
%%exposure were assessed. For each enzymes in the fungal
%%toluene-degrading pathway described in~\cite{Parales2008} at least one
%%ortholog was found. The genomic organization of these homologs in {\claImm} is
%%highly interesting since five gene clusters involved in toluene
%%degradation were identified. In the case of the subpathway of catechol
%%oxidation, 4 out of 6 genes are clustered. While functional clusters
%%are common in fungi~\cite{Cooper1996,Hittinger2004,Keller1997}, it is
%%the first time that clusters related to toluene degradation are reported.  
%%
%%The comparison of the fungal and bacterial toluene-degradation
%%pathways allowed to identify 11 genes in {\claImm} corresponding to 9
%%bacterial genes involved in 8 reactions from \textit{P. putida} and
%%\textit{P. mendocina}~\cite{Parales2008}. The fact that the homologs
%%to bacterial enzymes involved in the toluene-degradation do not cover
%%the full catabolism, might be an evidence that in a 
%%toluene-contaminated environment {\claImm} and bacterial species might
%%be co-metabolizing toluene in a way similar to what was shown for
%%polyaromatic hydrocarbons~\cite{Peng2008}.  Among those 11 genes, four
%%were classified as horizontally transfered from bacteria by HGTfinder,
%%further underlining the close interplay between bacteria and the
%%studied \textit{Cladophialophora}. While HGT are common between fungi
%%and bacteria, they were only recently reported in the group of black
%%yeasts, i.e. for {\exoDer}~\cite{Chen2014}. Especially interesting are
%%CLAIMM\_02205 and CLAIMM\_02199, which apparently were transfered as a
%%cluster. CLAIMM\_02205 is the only gene in the genome of {\claImm}
%%that can convert Protocatechuate to Catechol. CLAIMM\_02199 is the
%%only gene in {\claImm} that can catalyze the hydration of
%%2-oxopent-4-eneoate into 4-hydroxy-2-oxovalerate. It is also
%%interesting to observe that relation between degradation and
%%opportunistic pathogenicity also holds in bacteria.
%%
%%While {\claImm} seems to have the enzymes necessary to degrade
%%toluene, the use of this compound as sole carbon-source drives the
%%fungus into an energy-saving state. Core processes
%%like cellular amid metabolism, cellular respiration, energy-coupled
%%proton transport, organic substance metabolism are negatively affected
%%by the xenobiotics. Translation  is one of the most enriched GO term
%%(fdr=\E{1.17}{46}) in the set of downregulated genes. This fits well
%%with the repression of 17 tRNAs during toluene exposure. In
%%contrast snoRNAs represent the only type of basal ncRNAs that are significantly
%%upregulated in the {\tol} experiment. SnoRNAs play a key role as
%%chaperone in the folding of mature rRNA, by binding to the
%%ribosome~\cite{Watkins2012}. In \textit{Eschericiae coli}, toluene
%%triggers the rapid disaggregation of ribosomes, either by denaturing
%%the toluene or by triggering the release of rRNA degrading
%%enzyme~\cite{Jackson1965}. Here the increased expression of snoRNAs
%%might be interpreted as response to the destabilizing effect of
%%toluene on the structure of mature or nascent rRNAs.
%%
%%Chaperones HSP20, HSP40 and HSP70, which are markers of
%%cellular stress~\cite{Lindquist1988}, are either highly expressed or
%%highly upregulated in the {\tol}-experiment. This fact, together with
%%the overrepresentation of proteins involved in the transport of
%%misfolded proteins, might indicate that toluene is having a negative
%%impact on the protein-folding process. A similar protein-damage
%%response was recently reported in the toluene degrading bacteria
%%\textit{Pseudomonas putida S12}~\cite{Volkers2015}. Toluene exposure
%%also lead to the overexpresion of genes harboring protein domains
%%linked to DNA repair, like ku70/ku80 or rad14, indicating that DNA is
%%being {\claImm} is damaged. This is in line with the recognized
%%mutagenic properties of aromatic and other poly-aromatic
%%compounds~\cite{Munoz2011,Singer1996}. 
%%
%%Toluene further triggers the expression of genes involved in the
%%degradation of xenobiotic and detoxification, like Glyoxylase II and
%%lutathione S-transferase as well as antioxidants, like ascorbate
%%peroxidase carotenoid oxygenase. HppD, a key enzyme in the production
%%of Pyomelanin, one of three types of melanin produced by the fungus,
%%is upregulaed by a factor 79 when {\claImm} is exposed to toluene. It
%%is the same type of melanin that was shown to be overexpressed when
%%{\exoDer} is grown on skin~\cite{Poyntner2016}. Melanin is known to
%%efficiently protect cells agains reactive oxygen
%%species~\cite{Sichel1991}. Pyomelanin was recently shown to provide a
%%considerable tolerance to hydrogen peroxide stress~\cite{Ahmad2016} in
%%\textit{Ralstonia solanacearum}.
%%
%%Six genes belonging to the stress-induced MAPK pathway in fungi are
%%upregulated in the {\tol} experiment. This signaling pathways is
%%related to various kind of stress, among other oxidative
%%stress~\cite{Bahn2007}. The overexpression of genes related to the
%%MAPK pathway, damaged-DNA repair and melanin production strongly
%%indicate that toluene is causing an oxidative burden on {\claImm}.
%%This observation has already been reported in bacteria, where it is
%%hypothesized that the solvent impede the correct electron transport,
%%leading to an increase of reactive oxygen
%%species~\cite{Dominguez2006}.
%%
%%Toluene increase cell wall fluidity in
%%bacteria~\cite{Sikkema1995} while the exposure of {\sacCer} to ethanol
%%was shown to have an impact on the composition of the lipid
%%bilayer~\cite{Henderson2014}. In {\claImm}, 16 genes involved in
%%chitin synthase, regulation,  modification, degradation as well as
%%1.3-$\beta$-glucan synthesis and processing are repressed. This might
%%be explained by the reduced growth rate of the fungus in toluene
%%compared to glucose. {\claImm} ergosterol biosynthetic pathway exhibits
%%three downregulated genes (Erg2, Erg5, Erg10) while Erg12, mevalonate
%%kinase, is upregulated (See Supplementary Table 1). Still the content
%%ergosterol in the fungus might increase through hydrolasation of
%%ergosteryl oleate, since a homolgue of sterol esterase (TGL1) is
%%upregulated by a factor 5 in the {\tol} experiment. This might help
%%{\claImm} to stabilize its membrane, as it was previously shown that
%%ergosterol composition of the lipid bilayer are highly correlated with
%%ethanol resistance~\cite{Aguilera2006}.
%%
%%In conclusion, although \textit{C.immunda} shows strong evidence of
%%cellular stress in presence of the toluene concentration used in our  
%%experiments, the fungus is still one of the most efficient fungi in
%%the complete mineralization of toluene~\cite{Blasi2016}. The
%%hypothesis that the toluene degrading enzymes might be responsible of
%%the success of black yeasts as pathogens due to the common chemical
%%nature of toluene and neurotransmitters needs still to be
%%demonstrated. The functional characterization of target mutants in
%%those genes could help in shading light in this open question of black
%%yeast ecology and pathogenesis.

%%Nevertheless,
%%its pathogenic nature impairs its direct use in biofilters.


%%While {\claImm} seems to have the enzymes necessary to
%%degrade toluene, the exposure to this xenobiotic drives the fungus
%%into a stress state, leading to a dramatic transcriptome shift.






\text{TODO} NEED TO WRITE A CONCLUSION PARAGRAPH

%%
%%
%%
%%
%%
%%along the
%%stress-activated pathway 
%%
%%Inversely, {\claImm}
%%shuts down large part of the basal metabolism
%%
%%\TODO{Trichothecene}
%%Still all the other enzymes involved in
%%the production of these three mycotoxins, 
%%i.e. Tri7, Tri8, Tri11, Tri101, Tri13, and Tri16  are present in the
%%genome of {\claImm}. It should be noted that in {\claImm}, in contrast
%%to other trichothecene-producing fungi, the trichothecene-related
%%enzymes are not clustered. A screening for mycotoxin produced by
%%{\claImm} did not find any. Moreover none of the identified
%%trichothecene-related genes were significantly regulated upon growth
%%with toluene.
%%
%%
%%\TODO{This may indicate
%%that {\claImm} can to produce trichothecene but not
%%trichothecene-derived toxins.}








%%Total number of transcript
%%cat exoDer.lncRNA.gff3 | perl ~/bin/addGeneToNCPasaAssemblies.pl >
%%temp
%%grep -P "\tmRNA\t" -c temp
%%perl  ~/bin/convertOvelappingGeneToSpliceVariant.pl -f temp -g temp  | sed -r 's/CDS/exon/' > exoDer.lncRNAFinal.gff3
%%Total number of genes
%%perl ~/bin/convertOvelappingGeneToSpliceVariant.pl -f temp -g temp |
%%grep -Pc '\tgene\t'
%%


%%\begin{center}
%%  \begin{figure}
%%    \includegraphics[width=\linewidth]{./geneEnrichment.eps}
%%    \caption{\label{fig:geneEnrichment}}
%%  \end{figure}
%%\end{center}




%%\section{results}
%%\subsection{RNA-Seq-based genome re-annotation}
%%\subsubsection{Transcript assembly}
%%In order to better understand the transcriptome response of \exoDer,
%%fungal cells were cultured on \TODO{...} and \TODO{...}. Three
%%biological replicates were sequenced for both experimental
%%conditions (Skin/Control). A total of 155 mio reads were mapped to the
%%genome. A PCA showed that ~\ref{fig:}
%%
%%The current transcript annotation of \exoDer was updated by using RNA-Seq data.
%%Transcripts were reconstructed either ab-initio with Trinity~\cite{Haas2013}
%%or based on the alignment of reads with CUFFLINKS~\cite{Trapnell2012}
%%and Trinity. PASA~\cite{Haas2003} was then used to 1) generate a
%%comprehensive set of transcripts 2) to update the genome annotation
%%of \exoDer 
%%%(see Figure~\ref{fig:Workflow}) 
%%3) detect new coding and non-coding transcripts.
%%
%%Out of the 9577 protein coding transcripts annotated in \exoDer, 3391 were
%%not modified by PASA while 4168 new splice variants were detected (See
%%Supplementary Table S1). In 18 cases two neighboring transcripts were
%%merged into one (See Supplementary Table S1).
%%Besides the PASA-predicted alternative splice variants, 2284 new
%%transcripts 
%%%grep -Po "\tmRNA\t" exoDer.newProtein.gff3 -c 
%%that originates from 871 loci 
%%%grep -Po "\tmRNA\t" exoDer.newProtein.gff3 -c 
%%and that codes for proteins of at least 100 amino acids were predicted
%%by Transdecoder~\cite{Haas2013} (see Supplementary Table S1).  No CDS
%%from the newly annotated coding transcript overlapped with any of the
%%annotated ncRNAs.  
%%
%%%% snakemake -d `pwd` -s `pwd`/Reannotation.Snakemake --stats snakemake.stats -j 100 exoDer.lncRNA.gff3 
%%
%%Further we looked at the presence of new non-coding transcripts by
%%fetching all PASA assembled transcripts 1) that did not overlap with
%%the updated protein coding-annotation, 2) that did not contain a
%%pfam-domain) 3) that were not reported as coding by CPAT~\cite{Wang2013}
%%(coding p-value$lt$0.001) and 4) that did not show sequence homology
%%to SwissProt (Blastp p-value$gt$0.001). A total of 7778 non-coding
%%transcripts 
%%%cat exoDer.lncRNA.gff3 | perl ~/bin/addGeneToNCPasaAssemblies.pl | grep -P "\tmRNA\t" -c 
%%%%
%%coming from 4956 loci 
%%%%perl ~/bin/addGeneToNCPasaAssemblies.pl < reAnnotation/exoDer.lncRNA.gff3 > temp && perl ~/bin/mapTranscriptToLocus.pl -f temp -g ../exoDer.fasta | sed -r 's/m.//' | fgrep -wf - temp | grep gene > exoDer.lncRNA.locus.gff3 && wc -l exoDer.lncRNA.locus.gff3
%%were assembled (see Supplementary
%%Table S1). Among them 40 overlapped with the set of de-novo predicted
%%non-coding RNA.  
%%%% snakemake -d `pwd` -s `pwd`/Reannotation.Snakemake --stats snakemake.stats -j 100 exoDer.noKnownFunc.gff3 
%%Finally a total of 6630 transcripts 
%%%% cat exoDer.noKnownFunc.gff3 | perl ~/bin/addGeneToNCPasaAssemblies.pl | grep -P "\tmRNA\t" -c 
%%from 5769 loci were assembled that
%%%% cat exoDer.noKnownFunc.gff3 | perl ~/bin/addGeneToNCPasaAssemblies.pl | perl ~/bin/mapTranscriptToLocus.pl -f /dev/stdin -g ../../exoDer.fasta | sort -u | wc -l
%%could neither be considered as coding by Transdecoder nor classified
%%as non-coding. 
%%
%%\subsubsection{Conserved Elements}
%%We had a deeper look at conserved coding and non-coding elements. To
%%this aim the genomes of 5 pathogenic fungi (\exoDer, \claImm, \fonPed,
%%\horWer and \canAlb) were aligned with multiz~\cite{Blanchette:04a}.
%%RNAz~\cite{Washietl2005} and RNAcode~\cite{Washietl2011} were
%%then run on the multiple genomes alignment in order to detect non-coding
%%and coding elements. A total of 11168 conserved regions covering
%%58$\%$ of the genome were returned by the alignment pipeline. RNAz returned a
%%total of 895 conserved non-coding loci with a P-score $> 0.9$. 683
%%hits overlapped 681 protein coding genes (348 UTR, 711 UTR,and 348
%%RNAz loci overlapped with UTRs. 182 elements overlapped with 
%%non-coding transcripts, 30 with annotated non-coding RNAs and 89 hits
%%were intergenic. Because RNAz hits have no strand information a single
%%hit might overlap to transcript located on both strands.
%%
%%RNAcode predicted 24613 conserved coding elements with a p-value
%%$<1e^{-3}$. Among them 23200 (94.2$\%$) mapped to annotated proteins,
%%312 mapped to non-coding transcripts and 150 mapped to the list of
%%transcripts that could not be strictly classified as coding or
%%non-coding. 
%%
%%\subsubsection{Novel Protein Coding genes}
%%With the help of the transcriptomic data and the previoulsy published
%%genome annotation~\cite{Chen2014} a set of 871 new protein-coding
%%loci were annotated in the genome. Given the genome size of \exoDer
%%the total number of 10436 protein-coding loci is in-line with that of
%%of \exoMes(29Mb,9121loci) , \exoSid(29Mb,10114 loci) and \exoSpi(32MB,
%%12049 loci). 
%%%% Loci new Protein
%%%% cat exoDer.newProtein.gff3 | perl ~/bin/mapTranscriptToLocus.pl -f
%%%% /dev/stdin -g ../exoDer.fasta | sort -u | fgrep -wf -
%%%% exoDer.newProtein.gff3  > exoDer.newProtein.locus.gff3
%%
%%%%Get Protein sequence
%%%%gff3_file_to_proteins.pl exoDer.newProtein.gff3 ../exoDer.fasta |
%%%%cat -v | sed -r 's/\^\\/ /g' > exoDer.newProtein.faa
%%%%
%%
%%%%BLAST coding against nr
%%%%blastall -p blastp -d /media/work/share/database/nr -i exoDer.newProtein.faa -a 8 -m 8 -v 1 -b 1 >NewPblastpRes.csv
%%%%cat NewPblastpRes.csv | perl -lane 'if($F[10]<=0.001){print}' | cut -f 1 | sort -u  | wc -l 
%%%%cat NewPblastpRes.csv | perl -lane 'if($F[10]<=0.001){print}' | cut -f 1 | sort -u  | fgrep -wf - exoDer.newProtein.gff3 | perl ~/bin/mapTranscriptToLocus.pl -f /dev/stdin -g ../../exoDer.fasta | sort -u  | wc -l
%%
%%%%bedtools intersect -a ../exoDer.newProtein.locus.gff3 -b ../exoDer.ConservedRegion.bed -wa -wb | grep gene | cut -f 9 | sort -u > listOfGeneInConservedRegions.csv 
%%684 gene loci overlapped with a genomic regions contained in the multiple genomes alignment. 
%%The 2284 predicted protein-coding transcripts were also blasted against the nr-database. 
%%578 proteins from 131 loci had a hit with a p-value $lt$ 0.001. 92 out
%%of the 131 loci overlapped with regions of the multiple genomes
%%alignment.
%%%%cat ../exoDer.merged.tsv | grep asmbl | cut -f 1 | sort -u | wc -l
%%%%cat ../exoDer.merged.tsv | grep asmbl | cut -f 1 | sort -u | fgrep -wf - ../exoDer.newProtein.locus.gff3 | grep gene| cut -f 9 | sort -u | wc -l 
%%387 proteins from 133 loci were functionally annotated
%%
%%\TODO{Caro:check virtulence factors um gezielt nach der diffExp von
%%  ausgewaehlten proteinen zu suchen}
%%
%%\subsubsection{New Non-Coding Transcripts}
%%A total of 6630 new transcripts of unknown type (NNT)  was
%%identified. 
%%%%cat reAnnotation/exoDer.noKnownFunc.gff3 | perl -lane 'print $F[4]-$F[3]' | sort -k 1,1gr > nKFLength.csv
%%%%
%%Their sizes ranged from 218 nts to 8440 nts with a median length of
%%923nts and 90$\%$ of the transcripts having a length $lt$ 2470 nts.  
%%Interestingly three snoRNAs homolog to snosnR55, snosnR61 and
%%SNORD14 are located in introns of a single NNT (see
%%figure~\ref{fig:snoRNALncRNA}). 
%%
%%\begin{figure}
%%  \includegraphics[width=0.4\linewidth]{./figures/snoRNAInLncRNAIntron.pdf}
%%  \includegraphics[width=0.4\linewidth]{./figures/snoRNAInLncRNAIntronII.pdf}
%%  \caption{\label{fig:snoRNALncRNA}}
%%\end{figure}
%%
%%\begin{figure}
%%  \includegraphics[width=0.7\linewidth]{./figures/longerU4.pdf}
%%  \caption{\label{fig:longU4}}
%%\end{figure}
%%
%%%%bedtools intersect -a exoDer.lncRNA.locus.gff3 -b exoDer.RNAz.gtf -wa -wb  | cut -f 18 | sort -u | wc -l 
%%%%bedtools intersect -a exoDer.lncRNA.locus.gff3 -b exoDer.RNAz.gtf -wa -wb  | cut -f 9 | sort -u | wc -l 
%%A total of 3515 NNT-encoding loci overlapped with conserved
%%regions. Among them 186 overlapped with 195 RNAz predictions. 11 NNT-loci had at least two
%%RNAz hits. An example can be seen in figure~\ref{fig:lncRNARNAz}.
%%\begin{figure}
%%  \includegraphics[width=0.7\linewidth]{./figures/lncRNARNAz.pdf}
%%  \caption{\label{fig:lncRNARNAz}}
%%\end{figure}
%%
%%%%bedtools intersect -a exoDer.lncRNA.locus.gff3 -b exoDer.RNAcode.gtf -wa -wb  | cut -f 18 | sort -u | wc -l 
%%%%bedtools intersect -a exoDer.lncRNA.locus.gff3 -b exoDer.RNAcode.gtf -wa -wb  | cut -f 9 | sort -u | wc -l 
%%282 NNT-encoding loci overlapped with 426 RNAcode predictions. 36 NNT had at least two
%%RNAcode hits. %%An example can be seen in figure~\ref{fig:lncRNARNAz}.
%%%%\begin{figure}
%%%%  \includegraphics[width=0.7\linewidth]{./figures/lncRNARNAz.pdf}
%%%%  \caption{\label{fig:lncRNARNAz}}
%%%%\end{figure}
%%
%%\subsubsection{New Transcripts of Undetermined Type}
%%A total of 7778 new non-coding transcripts (NUT) was identified. Their sizes
%%ranged from 19 nts to 9192 nts with a median length of 307nts and
%%90$\%$ of the transcripts having a length $lt$ 891 nts. 
%%
%%%%Get a locus gff3 for the unknown function rRNA
%%%%perl ~/bin/addGeneToNCPasaAssemblies.pl < reAnnotation/exoDer.noKnownFunc.gff3 > temp && perl ~/bin/mapTranscriptToLocus.pl -f temp -g ../exoDer.fasta | sed -r 's/m.//' | fgrep -wf - temp | grep gene > exoDer.noKnownFunc.locus.gff3
%%12 de-novo predicted ncRNAs overlap with 10 NUT-locus. 
%%A single transcript overlaps four clustered snoRNAs that were
%%previously described in~\cite{Jochl2008}. The
%%transcription pattern indicates that snoRNAs might be expressed as a
%%polycistronic transcript that is subsequently spliced to deliver the
%%snoRNAs. 
%%
%%
%%
%%\subsection{Transcript Expression}
%%%%#Copy gtf files
%%%%cp exoDer.ncRNA.gtf exoDer.noKnownFunc.gtf exoDer.allProtein.gff3 exoDer.lncRNA.gtf exoDer.RNAz.gtf ../RNAseq/expressionLevel/
%%%%
%%%%#modify what has to be done
%%%%
%%%%#Prepare reference for rsem
%%%%rsem-prepare-reference --gtf all.transcripts.gtf --star --star-path /home/htafer/bin/ -p 8 ../../exoDer.fasta exoDer
%%%%
%%%%#### Start expression quantification ###
%%%%parallel -j 1 "rsem-calculate-expression --strand-specific -p 8 --star --star-path /home/htafer/bin {} exoDer {.}" ::: *.fastq
%%%%
%%%%##   Check most highly expressed  ##
%%%%sort -k 7,7gr Skin_S1.genes.results | head -n 100 > Skin_S1TopExpressed.csv
%%%%sort -k 7,7gr SkinControl_S1.genes.results | head -n 100 > SkinControl_S1TopExpressed.csv
%%%%sort -k 7,7gr WT371W.genes.results | head -n 100 > WT371WTopExpressed.csv
%%%%sort -k 7,7gr WT45C1W.genes.results | head -n 100 > WT45C1WTopExpressed.csv
%%%%##    Get enrichments ##
%%%%for i in *TopExpressed.csv; do cut -f 1 $i > $i.gene; done
%%%%parallel --dryrun -j 8 'j=`echo {1} | sed -r "s/exoDer.ann.//"`; enrichmentStat.R -b {1} -d {2} -t ${j} -D /media/work/share/database/InterProscanDictionary.dat' ::: exoDer.ann.* ::: *.gene
%%
%%Functional enrichment of the 100 most highly expressed
%%coding genes in the skin and control experiments was analysed. 
%%Under the control conditions, genes with domains related to Band 7
%%proteins (HMPREF1120\_07454, HMPREF1120\_01950, HMPREF1120\_06159) (fdr=0.001) and
%%Histone-related proteins (HMPREF1120\_06310, HMPREF1120\_01816,
%%HMPREF1120\_06252) were enriched. The 100 most highly expressed proteins in the skin
%%experiment were related to Heat shock protein (HSP) 70 (HMPREF1120\_02626,
%%HMPREF1120\_07756, HMPREF1120\_08142,fdr=0.002,3/5), phosphorylcholine
%%metabolism (HMPREF1120\_09233, HMPREF1120\_04356, fdr=0.004, 2/2) and
%%translation elongation (HMPREF1120\_08281, HMPREF1120\_0844, fdr=0.004,
%%2/2). Phosphorylcoline is used by a wide array of prokaryotic and
%%eukaryotic pathogens to hide from or modulate host responses to
%%infection~\cite{Clark2013,Clark2012}. In \canAlb two members of the HSP70
%%family are expressed on the cell surface and function as receptors for
%%antimicrobial peptides~\cite{Sun2010}.
%%
%%
%%
%%\subsection{Differential Expression}
%%\TODO{CARO}
%%
%%
%%
%%
%%
%%
%%
%%
%%
%%
%%

\ifcode
\renewcommand\listoflistingscaption{List of source codes}
\listoflistings
\fi


\bibliographystyle{plain}
%\bibliography{./temp}

\bibliography{./collection}


%----------------------------------------------------------------------------------------

\end{document}


